\chapter{Data Structures}\label{ch:data-structures}

Let a \define{dynamic set} $\vSomeSet$ be a mutable collection of distinct elements. Let $\vSomeElement$ be a pointer to an element in $\vSomeSet$ such that $\aSomeElementKey$ uniquely identifies the element in $\vSomeSet$. Let a \define{dictionary} be a dynamic set that supports insertion, deletion, and membership querying \citep{Cormen2022}.
%
\begin{itemize}
  \item $\cInsert[\vSomeSet, \vSomeElement]$ adds the element pointed to by $\vSomeElement$ to $\vSomeSet$.
  \item $\cDelete[\vSomeSet, \vSomeElement]$ removes the element pointed to by $\vSomeElement$ from $\vSomeSet$.
  \item $\cSearch[\vSomeSet, \vKey]$ returns a pointer $\vSomeElement$ to an element in the set $\vSomeSet$ such that $\aSomeElementKey \equals \vKey$; or $\nil$, if no such element exists in $\vSomeSet$.
  \item $\cMerge[\vSomeSet, \vSomeElement]$ adds the element pointed to by $\vSomeElement$, if no such element exists in $\vSomeSet$; otherwise, the result of applying a function to the element pointed to by $\vSomeElement$ and the existing element is added to $\vSomeSet$.
  \item $\cMaximum[\vSomeSet]$ returns a pointer $\vSomeElement$ to the maximum element of the totally ordered set $\vSomeSet$; or $\nil$ if $\vSomeSet$ is empty.
\end{itemize}