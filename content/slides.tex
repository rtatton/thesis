\documentclass[11pt]{beamer}

\usepackage{slides}

\begin{document}

\begin{frame}
	\maketitle
\end{frame}

\begin{section}{Introduction}

\begin{frame}{Digital Contact Tracing}
\begin{itemize}
  \item \define{Contact tracing}: a non-pharmaceutical intervention that aims to halt the spread of infectious disease by identifying and quarantining individuals that have been physically proximal with the infected \citep{PozoMartin2023}
  \pause
  \item \define{Digital contact tracing} (DCT): automates manual contact tracing by estimating physical interactions and notifying individuals of their infection risk \citep{Reichert2021}
\end{itemize}
\end{frame}

\begin{frame}{Digital Contact Tracing: Proximity Tracing}
\begin{figure}
  \centering
  \includegraphics[width=0.8\textwidth]{contact-discovery}
  \caption[Proximity tracing]{An approach to digital contact tracing that uses device-to-device communication to approximate in-person interaction \cite{Reichert2021}. While multiple protocols for proximity detection exist, Bluetooth Low Energy (BLE) is typically used because of its relative accuracy, energy efficiency, and broad support in mobile devices \citep{Shubina2020, Reichert2021}.}
\end{figure}
\end{frame}

\begin{frame}{Digital Contact Tracing: Broadcast Model}
\begin{figure}
  \centering
  \includegraphics[width=0.8\textwidth]{broadcast-model}
  \caption[Broadcast model]{A form of decentralized digital contact tracing in which an infected individual uploads their pseudonyms to a service that allows others to determine if they possess any of the pseudonyms belonging to that individual \cite{Reichert2021}.}
\end{figure}
\end{frame}

\begin{frame}{Digital Contact Tracing: Message-oriented Model}
\begin{figure}
  \centering
  \includegraphics[width=\textwidth]{messaging-model}
  \caption[Message-oriented model]{A form of decentralized digital contact tracing in which individuals exchange public keys to encrypt messages that are delivered to their contacts' postboxes \cite{Reichert2021}.}
\end{figure}
\end{frame}

\begin{frame}{Limitations and Considerations}
\begin{itemize}
  \item None incorporate both non-diagnostic information and indirect contacts to estimate infection risk
  \pause
  \item \citet{Cherini2023} propose exchanging pseudonyms of indirect contacts, but restrict themselves to diagnostic testing
  \pause
  \item \citet{Gupta2023} incorporate non-diagnostic information, but do not account for indirect contact
  \pause
  \item Accounting for indirect contact can substantially improve efficacy of DCT \citep{PozoMartin2023}
  \pause
  \item Privacy and security are paramount to adoption \citep{Oyibo2022, Afroogh2022}, key determinants of epidemic control \citep{PozoMartin2023}
\end{itemize}
\end{frame}

\begin{frame}{ShareTrace}
\begin{itemize}
  \item Developed in collaboration with Dataswyft during the COVID-19 pandemic \citep{Ayday2020}
  \pause
  \item Accounts for both non-diagnostic information and indirect contact to estimate infection risk
  \pause
  \item \citet{Ayday2021} describe a centralized/offline message-passing algorithm (``risk propagation'')
    \pause
    \begin{itemize}
      \item Centralized aggregation of personal data (e.g., PHI, contacts)
      \pause
      \item Delayed updates to exposure risk
      \pause
      \item Redundant communication and computation
    \end{itemize}
\end{itemize}
\end{frame}

\begin{frame}{Contributions}
\begin{itemize}
  \item Asynchronous formulation of risk propagation using the actor model, which permits a decentralized deployment
  \pause
  \item \define{Message reachability}: a generalization of temporal reachability that accounts for message-passing semantics on temporal networks
  \pause
  \item Classification of ShareTrace as mobile crowdsensing
  \pause
  \item Impact of parameter tuning on accuracy and efficiency
  \pause
  \item Exploratory data analysis
  \pause
  \item Reference implementation benchmarking
\end{itemize}
\end{frame}

\end{section}

\begin{section}{Proposed Design}

\begin{frame}{Asynchronous Risk Propagation: Assumptions}
\begin{itemize}
  \item Actor model \citep{Hewitt1973, Hewitt1977a, Hewitt1977b, Agha1985}
  \pause
  \item Arbitrary communication delays between actors
  \pause
  \item Personal data and contact pseudonyms are all stored locally
  \pause
  \item Messaging system may be centralized or decentralized
\end{itemize}
\end{frame}

\begin{frame}{Asynchronous Risk Propagation: Idea}
\begin{itemize}
  \item Risk scores and contacts are received by the actor as messages
  \pause
  \item Actor maintains exposure score history using an interval tree and contacts using a hash map
  \pause
  \item Data structures allow for delay between contact time, symptom reporting calculation, and message processing
  \pause
  \item Ensure finite message propagation and avoid uninformative risk score messages by associating a \define{send threshold} for each contact
  \pause
  \item \define{Send coefficient $\pSendCoefficient$} parametrizes the trade-off between accuracy and efficiency
\end{itemize}
\end{frame}

\end{section}

\begin{section}{Evaluation}

\begin{frame}{Evaluation}
\begin{itemize}
  \item Reference implementation using Akka-based actor model
  \item Random graphs: Barabasi-Albert \citep{Barabasi1999}, Erd\H{o}s-R\'{e}nyi \citep{Erdos1959}, Watts-Strogatz \citep{Watts1998}, and random regular \citep{Kim2003}
  \item How does the send coefficient affect accuracy and efficiency?
  \item How does the size of the contact network affect runtime?

\end{itemize}
\end{frame}

\begin{frame}{Experiment configurations}
\begin{table}
  \centering
  \small
  \renewcommand{\arraystretch}{2}
  \begin{tabular}{lcc}
    \toprule
    Aspect & Optimization & Benchmark \\
    \midrule
    Order $n$ & \qty{5}{K} & \qty{10}{K} -- \qty{100}{K} \\
    Size $m$ & \qty{50}{K} & \qty{1}{M} -- \qty{10}{M} \\
    Parameters & $\pSendCoefficient = \text{0.8 -- 2.0}$ & Defaults \\
    Distributions & $\{\text{Uniform}, \text{Normal}\}^3$ & Uniform \\
    Repetitions & 5 & 1 burn-in + 5 \\
    Networks & 160 (40 per type) per parameter & \num{2000} (500 per type) \\
    \bottomrule
  \end{tabular}
  \caption[Experiment configurations]{The notation $X^k$ is used to denote the $k$-ary Cartesian power of the set $X$. A ``burn-in'' repetition was used for Experiment 2 and Experiment 3 to avoid measuring the impact of Java class loading.}
\end{table}
\end{frame}

\begin{frame}{Results: Internetwork Accuracy}
\begin{figure}
  \centering
  \includegraphics[height=\squareFigHeight]{accuracy-percentiles}
  \caption[Cumulative accuracy distributions]{Internetwork accuracy distributions.}
\end{figure}
\end{frame}

\begin{frame}{Results: Intranetwork Accuracy}
\begin{figure}
  \centering
  \includegraphics[width=\wideFigWidth]{accuracy-proportions}
  \caption[Intranetwork accuracy distributions]{The dashed line inside each violin marks the median. The upper and lower dotted lines inside each violin mark the upper and lower quartiles, respectively.}
\end{figure}
\end{frame}

\begin{frame}{Results: Message-passing Efficiency}
\begin{figure}
  \centering
  \includegraphics[height=\squareFigHeight]{relative-receives}
  \caption[Message-passing efficiency]{The send coefficient $\pSendCoefficient = 1$ was used as a baseline for message-passing efficiency since it was found to be the maximum send coefficient that achieves perfect accuracy.}
\end{figure}
\end{frame}

\begin{frame}{Results: Correlation Coefficients}
\begin{figure}
  \centering
  \includegraphics[height=\squareFigHeight]{correlation}
  \caption[Correlation matrix of dataset attributes]{Each cell is the Spearman rank partial correlation coefficient \citep{Spearman1904}, controlling for the effect of the send coefficient. All coefficients are significant ($p < 0.01$), adjusting for multiple comparisons via the Holm–Bonferroni method \citep{Holm1979}.}
\end{figure}
\end{frame}

\begin{frame}{Results: Message-passing Runtime Regression}
\begin{figure}
  \centering
  \includegraphics[width=\wideFigWidth]{runtimes}
  \caption[Message-passing runtimes]{Message-passing runtimes.}
\end{figure}
\end{frame}

\begin{frame}{Results: Message-passing Runtime Regression}
\begin{figure}
  \centering
  \includegraphics[height=\squareFigHeight]{runtime-regression}
  \caption[Message-passing runtimes with regression lines]{Message-passing runtimes with quantile regression lines.}
\end{figure}
\end{frame}

\end{section}

\begin{section}{Conclusion}

\begin{frame}{Conclusion: Future Work}
\begin{itemize}
  \item Incorporate differential privacy techniques that are designed for DCT applications that utilize risk scores \citep{Romijnders2024}
  \pause
  \item Formally define the security and privacy characteristics of ShareTrace, using the framework proposed by \citet{Kuhn2021}
  \pause
  \item Conduct a simulation-based analysis of asynchronous risk propagation with COVI-AgentSim \citep{Gupta2020}
  \pause
  \item Explore the utility and feasibility of integrating decentralized technologies \citep{Troncoso2017, Trautwein2022, Shi2024, Keizer2024} and self-sovereign identity \citep{Preukschat2021} into the system design
\end{itemize}
\end{frame}

\begin{frame}[c]{ }
  \centering
  \Huge
  Thank you
\end{frame}

\end{section}

% A section is automatically created for references.
\begin{frame}[allowframebreaks]{References}
  \printbibliography
\end{frame}

\appendix

\begin{frame}{Digital Contact Tracing: Centralized Model}
\begin{figure}
  \centering
  \includegraphics[width=0.7\textwidth]{centralized-model}
  \caption[Centralized model]{A centralized entity is able to deanonymize an individual's pseudonyms in order to notify their contacts of infection risk \cite{Reichert2021}.}
\end{figure}
\end{frame}

\begin{frame}{Risk Propagation: Definitions}
\begin{itemize}
  \item \define{Risk score}, $\vScore_\vTime \in [0, 1]$: a timestamped infection probability where $\vTime \in \naturals$ is the time of its computation
  \item \define{Symptom score}: prior infection probability; accounts for an individual's demographics, symptoms, and diagnosis \citep{Briers2020, Menni2020}
  \item \define{Exposure score}: posterior infection probability; accounts for direct and indirect contact with others
  \item An individual's exposure score is derived by marginalizing over the joint infection probability distribution
    \begin{itemize}
      \item Naively computing this marginalization scales exponentially with the number of variables (i.e., individuals)
    \end{itemize}
\end{itemize}
\end{frame}

\begin{frame}{Risk Propagation: Definitions}
\begin{itemize}
  \item \define{Factor graph}: $\vGraph = (\vVariables, \vFactors, \vEdges)$
%  \pause
  \item \define{Variable \vertexName} $\vVariable: \eventSpace \rightarrow \{0, 1\} $: a random variable that represents the infection status of an individual, where
    \begin{equation*}
      \vVariable(\event) =
        \begin{cases}
          0 & \text{if } \event = \var{healthy} \\
          1 & \text{if } \event = \var{infected}
        \end{cases}
    \end{equation*}
%    \pause
    \item \define{Factor \vertexName} $\vFactor: \vVariables \times \vVariables \rightarrow [0, 1]$: the transmission of infection risk between two individuals
\end{itemize}

\begin{figure}
\centering
\begin{tikzpicture}[ampersand replacement=\&]
  \matrix[row sep=1.5em, column sep=0.75em] {
    \& \factor[minimum size=1em] {f12} {above:$\vFactor[12]$} {} {}; \&\&
    \factor[minimum size=1em] {f23} {above:$\vFactor[23]$} {} {}; \& \\
    \node[latent, minimum size=2em] (v1) {$\vVariable[1]$}; \&\&
    \node[latent, minimum size=2em] (v2) {$\vVariable[2]$}; \&\&
    \node[latent, minimum size=2em] (v3) {$\vVariable[3]$}; \\
  };
  \edge[-] {v1} {f12};
  \edge[-] {v2} {f12};
  \edge[-] {v2} {f23};
  \edge[-] {v3} {f23};
\end{tikzpicture}
\end{figure}
\end{frame}

\begin{frame}{Synchronous Risk Propagation: Definitions}
\begin{itemize}
  \item \define{Set of recent risk scores of \indexed{i}{individual}}:
    \begin{equation*}
      \vScores_i =\setBuilder{\vScore_t}{\vReferenceTime - \vTime < \pScoreExpiry} \in \vScores
    \end{equation*}
%    \pause
    \item \define{Contact set}:
      \begin{equation*}
        \vContacts = \setBuilder{(i, j, \vTime)}{i \neq j, \vReferenceTime - \vTime < \pContactExpiry}
      \end{equation*}
      $(i, j, \vTime)$ is the \emph{most recent} contact between $i$ and $j$
%      \pause
      \item \define{Reference time} $\vReferenceTime \in \naturals$: defines relevance of inputs
%      \pause
      \item Risk scores and contacts have finite relevance: $\pScoreExpiry, \pContactExpiry \in \naturals$
\end{itemize}
\end{frame}

\begin{frame}{Synchronous Risk Propagation: Pseudocode}
\begin{function}{\nRiskPropagation}[\vScores, \vContacts]
  \State{$\vRiskScores{i}{n - 1} \assign \topK{\vScores_i}$}
%  \pause
  \State $\vExposureScore{i}{n - 1} \assign \max \vRiskScores{i}{n - 1}$
%  \pause
  \State $\vExposureScore{i}{n} \assign \infty$
%  \pause
  \While{$\| \vExposureScores{n} - \vExposureScores{n - 1} \| > \pDiff$}
%  \pause
    \State $\vVariableMessage{i}{j}{n} \assign \vRiskScores{i}{n - 1} \setminus \setBuilder{\vFactorMessage{j}{i}{\ell}}{\ell \in \intInterval{1}{n - 1}}$
%    \pause
    \State $\vFactorMessage{i}{j}{n} \assign \max \setBuilder{\pTransmissionRate \vScore_\vTime}{\vScore_\vTime \in \vVariableMessage{i}{j}{n}, \vTime < \vTime_{ij} + \pTimeBuffer}$
%    \pause
    \State $\vRiskScores{i}{n} \assign \topK{\setBuilder{\vFactorMessage{j}{i}{n}}{\vFactor[ij] \in \vNeighbors_i}}$
%    \pause
    \State $\vExposureScore{i}{n} \assign \max \vRiskScores{i}{n}$
%    \pause
  \EndWhile
  \State \Return $\vExposureScores{n}$
\end{function}
\end{frame}

\begin{frame}{Synchronous Risk Propagation: Limitations}
\begin{itemize}
  \item Centralized aggregation of personal data (e.g., PHI, contacts)
%  \pause
  \item Delayed updates to exposure risk
%  \pause
  \item Redundant communication and computation
\end{itemize}
\end{frame}

\begin{frame}{Asynchronous Risk Propagation: Assumptions}
\begin{itemize}
  \item Actor model \citep{Hewitt1973, Hewitt1977a, Hewitt1977b, Agha1985}
%  \pause
  \item Arbitrary communication delays between actors
%  \pause
  \item Personal data and contact pseudonyms are all stored locally
%  \pause
  \item Messaging system may be centralized or decentralized
\end{itemize}
\end{frame}

\begin{frame}{Asynchronous Risk Propagation: Pseudocode}
\begin{function}{\nCreateActor}
  \State $\aActorContacts \assign \emptyset$
  \State $\aActorScores \assign \emptyset$
  \State $\aActorExposure \assign \cNullRiskScore$
  \State \Return $\vActor$
\end{function}
\begin{function}{\nNullRiskScore}
  \State $\aScoreValue \assign 0$
  \State $\aScoreTime \assign 0$
  \State \Return $\vScore$
\end{function}
\end{frame}

\begin{frame}{Asynchronous Risk Propagation: Pseudocode}
\begin{function}{\nRiskScoreTtl}[\vScore]
  \State \Return $\pScoreExpiry - (\vReferenceTime - \aScoreTime)$
\end{function}
\begin{function}{\nContactTtl}[\vContact]
  \State \Return $\pContactExpiry - (\vReferenceTime - \aContactTime)$
\end{function}
\end{frame}

\begin{frame}{Asynchronous Risk Propagation: Pseudocode}
\begin{function}{\nHandleRiskScore}[\vActor, \vScore]
  \If{$\cRiskScoreTtl[\vScore] > 0$}
%    \pause
    \State $\aScoreKey \assign [\aScoreTime, \aScoreTime + \pScoreExpiry)$
%    \pause
    \State $\cMerge[\aActorScores, \vScore]$
%    \pause
    \State $\cUpdateExposureScore[\vActor, \vScore]$
%    \pause
    \ForEach{$\vContact \in \aActorContacts$}
      \State $\cApplyRiskScore[\vActor, \vContact, \vScore]$
    \EndFor
  \EndIf
\end{function}
\end{frame}

\begin{frame}{Asynchronous Risk Propagation: Pseudocode}
\begin{function}{\nUpdateExposureScore}[\vActor, \vScore]
  \If{$\aActorExposureValue < \aScoreValue$}
%    \pause
    \State $\aActorExposure \assign \vScore$
%  \pause
  \ElsIf{$\cRiskScoreTtl[\aActorExposure] \leq 0$}
%    \pause
    \State $\aActorExposure \assign \cMaximum[\aActorScores]$
  \EndIf
\end{function}
\end{frame}

\begin{frame}{Asynchronous Risk Propagation: Pseudocode}
\begin{function}{\nApplyRiskScore}[\vActor, \vContact, \vScore]
  \If{$\aContactTime + \pTimeBuffer > \aScoreTime$}
%    \pause
    \State $\aNewScoreValue \assign \pTransmissionRate \cdot \aScoreValue$
%    \pause
    \State $\cSend[\aContactName, \vNewScore]$
  \EndIf
\end{function}
\end{frame}

\begin{frame}{Asynchronous Risk Propagation: Pseudocode}
\begin{function}{\nSetSendThreshold}[\vContact, \vScore]
  \State $\aNewScoreValue \assign \pSendCoefficient \cdot \aScoreValue$
%  \pause
  \State $\aContactThreshold \assign \vNewScore$
\end{function}
\end{frame}

\begin{frame}{Asynchronous Risk Propagation: Pseudocode}
\begin{function}{\nUpdateSendThreshold}[\vActor, \vContact]
  \If{$\aContactThresholdValue > 0$}
%    \pause
    \If{$\cRiskScoreTtl[\aContactThreshold] \leq 0$}
%      \pause
      \State $\vScore \assign \cMaximumOlderThan[\aActorScores, \aContactTime + \pTimeBuffer]$
%        \pause
      \State $\aNewScoreValue \assign \pTransmissionRate \cdot \aScoreValue$
%        \pause
      \State $\cSetSendThreshold[\vContact, \vNewScore]$
    \EndIf
  \EndIf
\end{function}
\end{frame}

\begin{frame}{Asynchronous Risk Propagation: Pseudocode}
\begin{function}{\nApplyRiskScore}[\vActor, \vContact, \vScore]
  \State $\cUpdateSendThreshold[\vActor, \vContact]$
%  \pause
  \If{$\aContactThresholdValue < \aScoreValue \AND \aContactTime + \pTimeBuffer > \aScoreTime$}
%    \pause
    \State $\aNewScoreValue \assign \pTransmissionRate \cdot \aScoreValue$
%    \pause
    \State $\cSetSendThreshold[\vContact, \vNewScore]$
%    \pause
    \State $\aContactBuffered \assign \vNewScore$
  \EndIf
\end{function}
\end{frame}

\begin{frame}{Asynchronous Risk Propagation: Pseudocode}
\begin{function}{\nHandleFlushTimeout}[\vActor]
  \ForEach{$\vContact \in \aActorContacts$}
%    \pause
    \If{$\aContactBuffered \notEquals \nil$}
%      \pause
      \State $\cSend[\aContactName, \aContactBuffered]$
%      \pause
      \State $\aContactBuffered \assign \nil$
    \EndIf
%    \pause
    \If{$\cContactTtl[\vContact] \leq 0$}
      \State $\cDelete[\aActorContacts, \vContact]$
    \EndIf
  \EndFor
\end{function}
\end{frame}

\begin{frame}{Asynchronous Risk Propagation: Pseudocode}
\begin{function}{\nHandleContact}[\vActor, \vContact]
  \If{$\cContactTtl[\vContact] > 0$}
%    \pause
    \State $\aContactThreshold \assign \cNullRiskScore$
%    \pause
    \State $\aContactBuffered \assign \nil$
%    \pause
    \State $\aContactKey \assign \aContactName$
%    \pause
    \State $\cMerge[\aActorContacts, \vContact]$
%    \pause
    \State $\vScore \assign \cMaximumOlderThan[\aActorScores, \aContactTime + \pTimeBuffer]$
%    \pause
    \State $\cApplyRiskScore[\vActor, \vContact, \vScore]$
  \EndIf
\end{function}
\end{frame}

\begin{frame}{Results: Message Reachability}
\begin{figure}
  \centering
  \includegraphics[height=\squareFigHeight]{message-reachability}
  \caption[Message reachability cumulative distributions]{Message reachability cumulative distributions.}
\end{figure}
\end{frame}

\begin{frame}{Results: Network Degree Distributions}
\begin{figure}
  \centering
  \includegraphics[height=\squareFigHeight]{network-degree-distributions}
  \caption[Contact network degree distributions]{All vertices in random regular contact networks had a degree of 20, so the distribution was omitted to provide more visual space for the distributions of other contact networks.}
\end{figure}
\end{frame}

%\begin{frame}{Results: Benchmarking Hypothesis Testing}
%\end{frame}

\begin{frame}{Prior Designs and Implementations}
\begin{itemize}
  \item  ``Thinking like a vertex'' with Apache Giraph
  \item Factor subgraph actors
  \item Driver-monitor-worker framework
  \item Projected subgraph actors \citep{Tatton2022b}
  \item Contact search
\end{itemize}
\end{frame}

\end{document}