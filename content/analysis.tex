\chapter{Evaluation}

\section{Reference Implementation}

A reference implementation of \cref{sec:asynchronous} is available on GitHub\footnote{\url{https://github.com/cwru-xlab/sharetrace-akka}}. Actors are implemented using the Akka toolkit\footnote{\url{https://doc.akka.io/docs/akka/2.8.5/typed/index.html}}, which offers high performance for large-scale actor systems. Experimental results indicate that the reference implementation can reliably handle contact networks with 1 million individuals and 10 million contacts, which makes it ideal for small-scale experiments. In addition to using the Akka toolkit, several other optimizations are implemented:
\begin{itemize}
  \item To reduce the size of event logs and result files, individual actor identifiers follow zero-based numbering and event records are serialized using the Ion format\footnote{\url{https://amazon-ion.github.io/ion-docs}} with shortened field names.
  \item To reduce memory usage, FastUtil\footnote{\url{https://fastutil.di.unimi.it}} data structures are used, including a specialized integer-based JGraphT\footnote{\url{https://jgrapht.org}} graph implementation \citep{Michail2020}. Also, singletons \citep{Gamma1995}, primitive data types, and reference equality are preferred where feasible and do not impact readability.
  \item To reduce runtime and increase throughput, logging is performed asynchronously with Logback\footnote{\url{https://logback.qos.ch/index.html}} and the LMAX Disruptor\footnote{\url{https://lmax-exchange.github.io/disruptor}}.
\end{itemize}

\Cref{fig:arrow-diagram} shows the dependencies among the application components. Contextualizing this implementation with prior implementations of the driver-monitor-worker (DMW) framework (see \cref{sec:dmw-framework}), \class{RiskPropagation} is the driver, \class{Monitor} is the monitor, and \class{User} is the worker. The key difference between this implementation and previous implementations of the DMW framework is that the workers are stateful, which is necessary for decentralization.
\begin{figure}[htbp]
\begin{equation*}
  \class{Main} \rightarrow \class{Runner} \rightarrow \class{RiskPropagation} \rightarrow \class{Monitor} \rightarrow \class{User} \rightarrow \class{Contact}
\end{equation*}
\caption[Arrow diagram of the reference implementation]{Arrow diagram of the reference implementation.}
\label{fig:arrow-diagram}
\end{figure}

\Cref{sec:asynchronous} describes the behavior of \class{User} and \class{Contact}. In order to evaluate \class{RiskPropagation}, each \class{User} also logs the following types of \class{UserEvent}:
\begin{itemize}
  \item \class{ContactEvent}: logged when the \class{User} receives an unexpired \class{ContactMessage}; contains the \class{User} identifier, the \class{Contact} identifier, and the contact time.
  \item \class{ReceiveEvent}: logged when the \class{User} receives an unexpired \class{RiskScoreMessage}; contains the \class{User} identifier, the \class{Contact} identifier, and the \class{RiskScoreMessage}.
  \item \class{UpdateEvent}: logged when the \class{User} updates its exposure score; contains the \class{User} identifier, the previous \class{RiskScoreMessage}, and the current \class{RiskScoreMessage}.
  \item \class{LastEvent}: logged when the \class{User} receives a \class{PostStop} Akka signal\footnote{\url{https://doc.akka.io/docs/akka/current/typed/actor-lifecycle.html\#stopping-actors}} after the \class{Monitor} has stopped; contains the \class{User} identifier and the time of logging the last event, besides \class{LastEvent}; used to detect the end time of message passing.
\end{itemize}
For reachability analysis, \class{RiskScoreMessage} contains the identifier of the \class{User} from which the message originated.

\class{Monitor} is an actor that is responsible for transforming the \class{ContactNetwork} into a collection of \class{User} actors and terminating when no \class{UpdateEvent} has occurred for a period of time. As with \class{User} actors, the \class{Monitor} logs several types of \class{LifecycleEvent}, the meanings of which should be self-explanatory:

\begin{multicols}{2}
\begin{itemize}
  \item \class{CreateUsersStart}
  \item \class{CreateUsersEnd}
  \item \class{SendRiskScoresStart}
  \item \class{SendRiskScoresEnd}
  \item \class{SendContactsStart}
  \item \class{SendContactsEnd}
  \item \class{RiskPropagationStart}
  \item \class{RiskPropagationEnd}
\end{itemize}
\end{multicols}

\class{RiskPropagation} logs execution properties, creates an Akka \class{ActorSystem} that creates a \class{Monitor} actor and sends it a \class{RunMessage}, and then waits until the \class{ActorSystem} terminates. Each execution of \class{RiskPropagation} is associated with a unique key that is included in each event record as mapped diagnostic context (MDC)\footnote{https://logback.qos.ch/manual/mdc.html}.

The \class{Runner} specifies how \class{RiskPropagation} is created and invoked, usually through some combination of statically defined behavior and runtime configuration.

Finally, \class{Main} is the entry point into the application. It is responsible for parsing \class{Context}, \class{Parameters}, and \class{Runner} from configuration and invoking \class{Runner} with \class{Context} and \class{Parameters} inputs. \class{Context} makes application-wide information accessible, such as the system time and user time\footnote{System time is always the wall-clock time and is included in each logged event record. User time is configurable to either be the wall-clock time or fixed at the reference time. The latter is ensures that no \class{RiskScoreMessage} and \class{ContactMessage} expires across executions of \class{RiskPropagation}.}, a pseudorandom number generator, \class{Runner} configuration, and loggers. \class{Parameters}, as the name suggests, is a collection of parameters that modify the behavior of the \class{Monitor}, \class{User}, and \class{Contact}.

In order to analyze the logs that were generated during the execution of \class{RiskPropagation}, they are transformed into a tabular dataset as follows:
\begin{enumerate}
  \item Load the execution properties for all executions of \class{RiskPropagation} that are associated with the same configuration.
  \item Process the stream of event records with one \class{EventHandler} per execution of \class{RiskPropagation}.
  \item Collect the results from each \class{EventHandler} and store them in a file.
  \item To analyze different configurations of \class{RiskPropagation}, load multiple result files and augment the results of each \class{RiskPropagation} execution with its execution properties.
  \item Flatten the resulting data structure and store the tabular dataset.
\end{enumerate}

For evaluation, the following event handlers were implemented:

\begin{itemize}
  \item \class{Reachability}: aggregates \class{ReceiveEvent}s that involve a distinct sender and receiver to determine the influence set cardinality, source set cardinality, and message reachability of each \class{User}.
  \item \class{Runtimes}: aggregates \class{LifecycleEvent}s and \class{LastEvent}s to determine the runtime of creating \class{User}s, sending \class{ContactMessage}s, sending \class{RiskScoreMessage}s, message passing, and the overall execution of \class{RiskPropagation}. Message passing runtime is the time elapsed from the start of sending \class{RiskScoreMessage}s until the last \class{LastEvent}.
  \item \class{UserEventCounts}: aggregates \class{UserEvent}s to determine the frequency of each subtype for each \class{User}.
  \item \class{UserUpdates}: aggregates \class{UpdateEvent}s to determine the new exposure score of each \class{User} and the change in value.
\end{itemize}

\subsection{Experimental Design}

The following research questions were the focus of evaluation:

\begin{enumerate}
  \item How do the send coefficient and tolerance affect the accuracy and efficiency of risk propagation?
  \item What is the runtime performance of risk propagation?
\end{enumerate}

Barabasi-Albert graphs \citep{Barabasi1999} are parametrized by the order $n$, the initial order $n_0$, and the size increase $m_0$ upon each incremental increase to the order. The latter two parameters are determined by solving \cref{eq:barabasi-albert-optimization}, where $\fracpart(x)$ is the fractional part of a real number $x$.
\begin{argmini}{n_0, m_0}{\fracpart(m_0)}{\protect\label{eq:barabasi-albert-optimization}}{}
  \addConstraint{n_0}{\in \intInterval{1}{n - 1}}
  \addConstraint{m_0}{\in \intInterval{1}{n_0}}
  \addConstraint{m_0}{= \frac{2m - n_0 (n_0 - 1)}{2(n - n_0)}}
\end{argmini}
Erd\"{o}s-R\'{e}nyi $G_{n,m}$ random graphs \citep{Erdos1959} are parametrized by the order $n$ and the size $m$. Random regular graphs \citep{Kim2003} are parametrized by the order $n$ and, using the degree sum formula, the degree $d = \lfloor \frac{2m}{n} \rfloor$. Lastly, Watts-Strogatz graphs \citep{Watts1998} are parametrized by the order $n$, the rewiring probability $p$ and the number of nearest neighbors $k = d + (d \bmod 2)$, which must be even.

\begin{table}
  \centering
  \begin{tabular}{ll}
    \toprule
    Parameter & Default value \\
    \midrule
    Seed & \num{12345} \\
    Transmission rate & \num{0.8} \\
    Send coefficient & \num{1} \\
    Tolerance & \num{0} \\
    Time buffer & \qty{2}{days} \\
    Risk score expiry & \qty{14}{days} \\
    Contact expiry & \qty{14}{days} \\
    Flush timeout & \qty{3}{seconds} \\
    Idle timeout & \qty{1}{minute} \\
    \bottomrule
  \end{tabular}
  \caption[Default parameter values for evaluation]{Default parameter values for evaluation.}
  \label{tab:default-parameters}
\end{table}

\begin{itemize}
  \item Fixed user time
  \item 5 contact networks with distinct risk scores and contact times
  \item Sampling procedure to generate dataset values: Given the probability density function $f_X$ and the cumulative distribution function $F_X$ of a random variable $X$, sample a value $x \sim f_X$ and evaluate $F_X(x)$.
\end{itemize}

Parmeter experiments:

\begin{itemize}
  \item $n = 10^4$, $m = 5 \cdot 10^4$
  \item Distributions: uniform, standard normal
  \item Send coefficients: 0.8–2.0, in increments of 0.1
  \item Tolerance: 0.001–0.01, in increments of 0.001
  \item All 9 distribution combinations: uniform, standard normal
  \item 5 contact networks with distinct risk scores and contact times
\end{itemize}

Runtime baseline experiment:

\begin{itemize}
  \item $n = 10^4$, $m = 10^5$
  \item All 9 distribution combinations: uniform, standard normal
  \item 1 burn-in + 5 contact networks with distinct risk scores and contact times
  \item Log lifecycle events and last event for message-passing runtime
\end{itemize}

Runtime experiment:

\begin{itemize}
  \item $n \in \setBuilder{10^5x}{x \in [1, 10]} \times m \in \setBuilder{10^6x}{x \in [1, 10]}$
  \item Uniform distribution for all 3 data types
  \item 1 burn-in + 5 contact networks with distinct risk scores and contact times
  \item Log lifecycle events and last event for message-passing runtime
\end{itemize}

%The value of \Cref{eq:reach} can be found by associating with each symptom score a unique identifier. If each actor maintains a log of the risk scores it receives, then the set of actors that receive the symptom score or a propagated risk score thereof can be identified. This set of actors defines the induced subgraph on which to compute \Cref{eq:reach} using a shortest-path algorithm \citep{Johnson1977}.

\subsection{Results}