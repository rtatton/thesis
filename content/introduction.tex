\chapter{Introduction}

% Outline
% 1. General COVID-19 stuff
% 2. Digital contact tracing and motivation for ShareTrace
% 3. Previous ShareTrace work and motivation for this work
% 4. Contributions of this work
% 5. Outline of thesis

Since the beginning of the COVID-19 pandemic, there has been a copious amount of research in mobile contact tracing solutions, most notably being the joint effort by Apple and Google \cite{AppleGoogle}. External reviews and surveys provide extensive comparison of existing solutions through the lenses of privacy, security, ethics, adversarial models, data management, scalability, interoperability, and more. References \cite{Ahmed2020} and \cite{Martin2020} provide thorough reviews of existing mobile contact tracing solutions with discussion of the techniques, privacy, security, and adversarial models. The former offers additional detail on the system architecture (i.e., centralized, decentralized, and hybrid), data management, and user concerns of existing solutions. Other notable reviews with similar discussion include \cite{Wen2020, Raskar2020, Cho2020, Dar2020, Lucivero2020}. Reference \cite{Kuhn2021} provides a formal framework for defining aspects of privacy for proximity-based contact tracing.

A number of online surveys have been conducted that examine user preferences of different aspects of contact tracing \cite{Simko2020, Altmann2020, Li2020}. A common finding across these surveys is that privacy and security continue to be of top concern for users, but contains some interesting nuance. For example, \cite{Altmann2020} surveyed over 10,000 individuals and found that there was over a 60-percent willingness to install a contact tracing mobile application. In a longitudinal study, \cite{Simko2020} found that user preferences regarding privacy were stable over time. Moreover, they found that users had fewer privacy concerns for proximity-based contact tracing, in comparison to location-based contact tracing, but that there was security concerns for proximity-based tracking. Contrary to much of the developed techniques that emphasize a decentralized approach, \cite{Li2020} observed that mobile contact tracing applications that implement a centralized design are significantly more likely to be installed at the country level. Additionally, they found that individuals are generally more comfortable with their location data and identity information accessible to health-, state-, and federal-level authorities, compared to application developers, and the general public.

A common design element across all of the aforementioned contact tracing methodologies is that they only consider direct interactions between users. While there are privacy benefits to this approach, a major limitation is that they cannot utilize information about indirect contact to more effectively reduce the spread of disease. ShareTrace addresses this limitation by constructing a factor graph and estimating infection risk via a message-passing algorithm. As such, this work labels the ShareTrace algorithm as \define{risk propagation}. The first work on ShareTrace was a white paper that focused on the motivation, design, and engineering details. Exclusive to \cite{Ayday2020} is a discussion on privacy, network roaming, protocol interoperability, and the usage of geolocation data. Furthermore, it includes detail on the system model and data flow. The second work on ShareTrace \cite{Ayday2021} formalizes the algorithmic details in a centralized setting and demonstrates its improved efficacy, compared to the framework developed by Apple and Google \cite{AppleGoogle}.

This work improves the efficiency of risk propagation and provides a concurrent, distributed, online, and non-iterative formulation using the actor model. To quantify the communication complexity of this new design, this work defines message reachability to account for the dynamics of message passing on a temporal network. 

The evaluation of risk propagation in this work entails: (1) the efficiency of risk propagation on both synthetic and real-world contact networks; (2) the scalability of risk propagation on synthetic contact networks; and (3) the accuracy of message reachability on synthetic and real-world networks. To keep the scope of this work focused, we defer to \cite{Ayday2021} on the privacy and security aspects of ShareTrace.

While message passing has been studied under specific epidemiological models \cite{Karrer2010, Li2021}, our formulation allows us to contextualize risk propagation as a novel usage of a contact network that does not require such assumptions to infer the transmission of disease. As a result, we introduce a form of reachability that can uniquely characterize the dynamics of message passing on a temporal graph. Our formulation of risk propagation aligns with its distributed extension, as introduced by \cite{Ayday2021}, which has connections to the actor model of concurrent computing \cite{Baker1977, Agha1986} and the ``think-like-a-vertex'' model of graph algorithms \cite{McCune2015}.