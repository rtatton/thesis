\chapter{Introduction}

\define{Contact tracing} is a non-pharmaceutical intervention that aims to halt the spread of infectious disease by identifying and quarantining individuals that have been physically proximal with the infected \citep{Brandt2022, PozoMartin2023}. In response to the pandemic of coronavirus disease 2019 (COVID-19) \citep{Zhu2020, Gorbalenya2020, Singh2021}, numerous approaches to \define{automatic contact tracing} (ACT) have been proposed.\citet{Shubina2020} provide an in-depth survey while \citet{Reichert2021} specifically focuses on techniques that utilize Bluetooth Low Energy (BLE) and provides a more current analysis of message-based approaches.

% Define decentralized digital contact tracing and how its been used to mitigate COVID-19

% Describe the efficacy of digital contact tracing and its shortcomings (privacy, security, widespread public adoption)

Effective contact tracing is inherently difficult because of the complex interactions between epidemiology, public health, ethics, politics, and sociocultural norms \citep{Brandt2022}.

% Improvements over other messaging-based approaches

Privacy risk of message-based approaches is that an adversary can monitor network traffic in order to reconstruct the contact network. Using a decentralized data layer would mitigate this attack mode.

%A common design element across all of the aforementioned contact tracing methodologies is that they only consider direct interactions between users. While there are privacy benefits to this approach, a major limitation is that they cannot utilize information about indirect contact to more effectively reduce the spread of disease. ShareTrace addresses this limitation by constructing a factor graph and estimating infection risk via a message-passing algorithm. As such, this work labels the ShareTrace algorithm as \define{risk propagation}. The first work on ShareTrace was a white paper that focused on the motivation, design, and engineering details. Exclusive to \citet{Ayday2020} is a discussion on privacy, network roaming, protocol interoperability, and the usage of geolocation data. Furthermore, it includes detail on the system model and data flow. The second work on ShareTrace \citep{Ayday2021} formalizes the algorithmic details in a centralized setting and demonstrates its improved efficacy, compared to the framework developed by Apple and Google \cite{AppleGoogle}.

% Contributions of this work

Higher-order contact tracing has demonstrated increased efficacy \citep{PozoMartin2023}

% ShareTrace prior work

% Contributions/organization of this work

% Proposed design
% Evaluation
% Conclusion
% Previous designs appendix

The remainder of this work is organized as follows. \Cref{ch:proposed-design} begins with the algorithmic foundations of ShareTrace: the risk propagation message-passing algorithm. Risk propagation is first described as a synchronous, offline algorithm, which is consistent with prior work \citep{Ayday2020, Ayday2021}. \Cref{ch:proposed-design} then provides an asynchronous formulation using the actor model, which permits a completely decentralized deployment of ShareTrace.

\Cref{ch:data-structures} and \cref{ch:pseudocode-conventions} respectively describe the data structures and pseudocode conventions that are used when specifying the algorithms in \cref{ch:proposed-design} and \cref{ch:previous-designs}.

%This work improves the efficiency of risk propagation and provides a concurrent, distributed, online, and non-iterative formulation using the actor model. To quantify the communication complexity of this new design, this work defines message reachability to account for the dynamics of message passing on a temporal network. 
%
%The evaluation of risk propagation in this work entails: (1) the efficiency of risk propagation on both synthetic and real-world contact networks; (2) the scalability of risk propagation on synthetic contact networks; and (3) the accuracy of message reachability on synthetic and real-world networks. To keep the scope of this work focused, we defer to \citep{Ayday2021} on the privacy and security aspects of ShareTrace.
%
%While message passing has been studied under specific epidemiological models \citep{Karrer2010, Li2021}, our formulation allows us to contextualize risk propagation as a novel usage of a contact network that does not require such assumptions to infer the transmission of disease. As a result, we introduce a form of reachability that can uniquely characterize the dynamics of message passing on a temporal graph. Our formulation of risk propagation aligns with its distributed extension, as introduced by \citet{Ayday2021}, which has connections to the actor model of concurrent computing \citep{Baker1977, Agha1986} and the ``think-like-a-vertex'' model of graph algorithms \citep{McCune2015}.

Related work \citep{Reichert2020, Cho2020, Cherini2023, Gupta2023}

Impact/efficacy of contact tracing \citep{PozoMartin2023}

Public opinion / ethics / adoption \citep{Oyibo2022, Afroogh2022, Simko2022}