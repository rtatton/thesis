\chapter{Introduction}

\define{Contact tracing} is a non-pharmaceutical intervention that aims to halt the spread of infectious disease by identifying and quarantining individuals that have been physically proximal with the infected \citep{PozoMartin2023}. To combat the pandemic of coronavirus disease 2019 (COVID-19) \citep{Zhu2020, Gorbalenya2020, Singh2021}, numerous approaches to \define{digital contact tracing} (DCT) have been proposed \citep{Shubina2020, Reichert2021, Troncoso2022, Gupta2023, Cherini2023}. While the implementation details vary considerably, all approaches to DCT aim to mitigate the spread of infection by automatically informing individuals of their infection risk.

A popular form of DCT is \define{proximity tracing}, which uses device-to-device (D2D) communication \citep{Haus2017} to approximate in-person interactions. While multiple protocols for proximity detection exist, Bluetooth Low Energy (BLE) is typically used because of its relative accuracy, energy efficiency, and broad support in mobile devices \citep{Shubina2020, Reichert2021}. Proximity tracing entails generating and exchanging pseudonyms (i.e., ephemeral identifiers, contact identifiers) between nearby mobile devices. What differentiates DCT applications is how these pseudonyms are generated and utilized. \define{Decentralized DCT} determines an individual's contacts and infection risk locally, thus avoiding the aggregation of sensitive personal data. \citet{Oyibo2022, Afroogh2022, Simko2022} all find that privacy and security are paramount to the adoption of DCT, which is a key determinant of achieving epidemic control \citep{PozoMartin2023}. While decentralization is not sufficient for strong privacy and security guarantees, it generally makes attaining such guarantees more feasible.

In the \define{broadcast model} of decentralized DCT, an infected individual uploads their information to a central service that allows others to determine if they possess any of the pseudonyms belonging to that individual \citep{Reichert2021}. A major limitation of the broadcast model is that an individual's infection risk does not account for indirect contacts, which can substantially improve the effectiveness of DCT \citep{PozoMartin2023}. Moreover, the broadcast model assumes an accurate diagnostic test exists and is broadly accessible and trusted by the public. To address the limitations of the broadcast model, \citet{Ayday2021} proposed ShareTrace, which uses a message-passing algorithm that incorporates non-diagnostic information and indirect contacts when estimating infection risk. However, the authors assume a centralized deployment, which exposes the entire contact network and personal data to a single entity.

Other approaches to \define{message-oriented DCT} \citep{Cho2020, Reichert2020} provide decentralization, but do not account for non-diagnostic information and indirect contacts. Recently, \citet{Cherini2023} proposed extending the broadcast model such that mobile devices also share the pseudonyms of their indirect contacts during D2D interactions, but still depend on diagnostic testing. \citet{Gupta2023} incorporate non-diagnostic information to proactively determine an individual's infection risk, but do not account for indirect contacts.

This work proposes an asynchronous formulation of the message-passing algorithm proposed by \citet{Ayday2021}, which permits a decentralized deployment of ShareTrace. In this way, this work addresses the limitations of other message-oriented DCT designs \citep{Cho2020, Reichert2020} and more recent works \citep{Cherini2023, Gupta2023} that do not incorporate both non-diagnostic information and indirect contacts when estimating infection risk. While message passing has been studied under specific epidemiological models \citep{Karrer2010, Li2021}, this work does not require such assumptions to infer the transmission of disease.

The remainder of this work is organized as follows. \Cref{ch:proposed-design} begins with the algorithmic foundations of ShareTrace: the risk propagation message-passing algorithm. Risk propagation is first presented as a synchronous, offline algorithm, which is consistent with prior work \citep{Ayday2021}. \Cref{ch:proposed-design} then provides an asynchronous formulation using the actor model. \Cref{ch:evaluation} evaluates a reference implementation of the proposed design. Various data distributions and contact network topologies are used to determine the parameter values that optimize asynchronous risk propagation for accuracy and efficiency. The runtime performance of the reference implementation is similarly evaluated. \Cref{ch:previous-designs} includes prior designs and implementations, including the approach proposed by \citet{Tatton2022b}. \Cref{ch:data-structures} and \cref{ch:pseudocode-conventions} respectively describe the data structures and pseudocode conventions that are used throughout this work. \Cref{ch:conclusion} concludes with directions for future work.