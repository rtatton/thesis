\section{Synchronous Risk Propagation}\label{sec:synchronous}

\newcommand{\pDiff}{\epsilon}
\newcommand{\topK}[1]{\text{top } K \text{ of } #1}
\newcommand{\vRiskScores}[2]{\vSet{R}_{#1}^{(#2)}}
\newcommand{\vExposureScore}[2]{r_{#1}^{(#2)}}
\newcommand{\vExposureScores}[1]{\mathbf{r}^{(#1)}}
\newcommand{\dist}{d}

\citet{Ayday2021} first proposed risk propagation as a synchronous, iterative message-passing algorithm that uses the factor graph to compute exposure scores. The first input to \cRiskPropagation{} is the set family $\vScores$, where
%
\begin{equation} \label{eq:score-set}
  \vScores_i =\setBuilder{\vScore_t}{\vRefTime - t < \pScoreExpiry} \in \vScores
\end{equation}
%
is the set of recent risk scores of \indexed{i}{individual}. The second input to \cRiskPropagation{} is the contact set
%
\begin{equation} \label{eq:contact-set}
  \vContacts = \setBuilder{(i, j, t)}{i \neq j, \vRefTime - t < \pContactExpiry}
\end{equation}
%
such that $(i, j, t)$ is the \emph{most recent} contact between \twoindexed{i}{j}{individual} that occurred from time $t$ until at least time $t + \pMinContactDuration$, where $\pMinContactDuration \in \naturals$ is the \define{minimum contact duration}\footnote{While \citet{Ayday2021} require contact over a $\pMinContactDuration$-contiguous period of time, the Centers for Disease Control and Prevention \citeyearpar{CDC2021} account for contact over a 24-hour period.}. Naturally, risk scores and contacts have finite relevance, so \eqref{eq:score-set} and \eqref{eq:contact-set} are constrained by the \define{risk score expiry} $\pScoreExpiry \in \naturals$ and the \define{contact expiry} $\pContactExpiry \in \naturals$, respectively. The \define{reference time} $\vRefTime \in \naturals$ defines the relevance of the inputs and is assumed to be the time at which \cRiskPropagation{} is invoked. For notational simplicity in \cRiskPropagation{}, let $\vSomeSet$ be a set. Then $\max \vSomeSet = 0$ if $\vSomeSet = \emptyset$.

\subsection{Variable Messages}

The current exposure score of \indexed{i}{individual} is defined as $\max \vScores_i$. Hence, a \define{variable message} $\vVariableMessage{i}{j}{n}$ from the variable node $\vVariable[i]$ to the factor node $\vFactor[ij]$ during \indexed{n}{iteration} is the set of maximal risk scores $\vRiskScores{i}{n - 1}$ from the previous $n - 1$ iterations that were not derived by $\vFactor[ij]$. In this way, risk propagation is reminiscent of the max-sum algorithm; however, risk propagation aims to maximize \emph{individual} marginal probabilities rather than the joint distribution \cite[pp. 411--415]{Bishop2006}.

\subsection{Factor Messages}

A \define{factor message} $\vFactorMessage{i}{j}{n}$ from the factor node $\vFactor[ij]$ to the variable node $\vVariable[j]$ during \indexed{n}{iteration} is an exposure score of \indexed{j}{individual} that is based on interacting with those at most $n - 1$ degrees separated from \indexed{i}{individual}. This population is defined by the subgraph induced in $\vGraph$ by
%
\begin{equation*}
  \setBuilder{v \in \vVariables \cap \vFactors \setminus \{\vVariable[j], \vFactor[ij]\}}{\dist(\vVariable[i], v) \leq 2(n - 1)},
\end{equation*}
%
where $\dist(u, v)$ is the distance between the nodes $u, v$. The computation of a factor message assumes the following.
%
\begin{enumerate}
  \item Contacts have a nondecreasing effect on an individual's exposure score.
  \item A risk score $\vScore_t$ is \define{relevant} to the contact $(i, j, t_{ij})$ if $t < t_{ij} +\pTimeBuffer$, where $\pTimeBuffer \in \naturals$ is a \define{time buffer} that accounts for delayed reporting of symptom scores and contacts. The expression $t_{ij} +\pTimeBuffer$ is called the \define{buffered contact time}.
  \item Risk transmission between contacts is incomplete. Thus, a risk score decays exponentially along its transmission path in $\vGraph$ at a rate of $\log \pTransmissionRate$, where $\pTransmissionRate \in (0, 1)$ is the \define{transmission rate}.
\end{enumerate}
%
To summarize, a factor message $\vFactorMessage{i}{j}{n}$ is the maximum relevant risk score in the variable message $\vVariableMessage{i}{j}{n}$ (or 0) that is scaled by the transmission rate $\pTransmissionRate$.

\citet{Ayday2021} assume that the contact set $\vContacts$ may contain (1) multiple contacts between the same two individuals and (2) \define{invalid} contacts, or those lasting less than $\pMinContactDuration$ time. However, these assumptions introduce unnecessary complexity. Regarding assumption 1, suppose \twoindexed{i}{j}{individual} come into contact $m$ times such that $t_k < t_\ell$ for $1 \leq k < \ell \leq m$. Let $\vFactorMessages_k$ be the set of relevant risk scores, according to the contact time $t_k$, where
%
\begin{equation*}
  \vFactorMessages_k = \setBuilder{\pTransmissionRate \vScore_t}{\vScore_t \in \vVariableMessage{i}{j}{n}, t < t_k + \pTimeBuffer}.
\end{equation*}
%
Then $\vFactorMessages_k \subseteq \vFactorMessages_\ell$ if and only if $\max \vFactorMessages_k \leq \max \vFactorMessages_\ell$. Therefore, only the most recent contact time $t_m$ is required to compute the factor message $\vFactorMessage{i}{j}{n}$. With respect to assumption 2, there are two possibilities.
%
\begin{enumerate}
  \item If an individual has at least one valid contact, then their exposure score is computed over the subgraph induced in $\vGraph$ by their contacts that define the neighborhood $\vNeighbors_i$ of the variable node $\vVariable_i$.
  \item If an individual has no valid contacts, then their exposure score is $\max \vScores_i$ or $0$, if all of their previously computed risk scores have expired.
\end{enumerate}
%
In either case, a set $\vContacts$ containing only valid contacts implies fewer factor nodes and edges in the factor graph $\vGraph$. Consequently, the complexity of \cRiskPropagation{} is reduced by a constant factor since fewer messages must be computed.

\subsection{Termination}

To detect convergence, the normed difference between the current and previous exposure scores is compared to the threshold $\pDiff \in \reals$. Note that $\vExposureScores{n}$ is the vector of exposure scores in the \indexed{n}{iteration} such that $\vExposureScore{i}{n}$ is \indexed{i}{component} of $\vExposureScores{n}$. The $\ell^1$ and $\ell^\infty$ norms are sensible choices for detecting convergence. \citet{Ayday2021} use the $\ell^1$ norm, which ensures that an individual's exposure score changed by at most $\pDiff$ after the penultimate iteration.

\begin{function}[H]{\nRiskPropagation}[\vScores, \vContacts]
  \State $(\vVariables, \vFactors, \vEdges) \assign \cCreateFactorGraph[\vContacts]$
  \State $n \assign 1$
  \ForEach{$\vVariable[i] \in \vVariables$}
    \State $\vRiskScores{i}{n - 1} \assign \topK{\vScores_i}$
    \State $\vExposureScore{i}{n - 1} \assign \max \vRiskScores{i}{n - 1}$
    \State $\vExposureScore{i}{n} \assign \infty$
  \EndFor
  \While{$\| \vExposureScores{n} - \vExposureScores{n - 1} \| > \pDiff$}
    \ForEach{$\{\vVariable[i], \vFactor[ij]\} \in \vEdges$}
      \State $\vVariableMessage{i}{j}{n} \assign \vRiskScores{i}{n - 1} \setminus \setBuilder{\vFactorMessage{j}{i}{k}}{k \in [1 \twodots n - 1]}$
    \EndFor
    \ForEach{$\{\vVariable[i], \vFactor[ij]\} \in \vEdges$}
      \State $\vFactorMessage{i}{j}{n} \assign \max \setBuilder{\pTransmissionRate \vScore_t}{\vScore_t \in \vVariableMessage{i}{j}{n}, t < t_{ij} + \pTimeBuffer}$
    \EndFor
    \ForEach{$\vVariable[i] \in \vVariables$}
      \State $\vRiskScores{i}{n} \assign \topK{\setBuilder{\vFactorMessage{j}{i}{n}}{\vFactor[ij] \in \vNeighbors_i}}$
    \EndFor
    \ForEach{$\vVariable[i] \in \vVariables$}
      \State $\vExposureScore{i}{n - 1} \assign \vExposureScore{i}{n}$
      \State $\vExposureScore{i}{n} \assign \max \vRiskScores{i}{n}$
    \EndFor
    \State $n \assign n + 1$
  \EndWhile
  \State \Return $\vExposureScores{n}$
\end{function}