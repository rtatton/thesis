\chapter{Introduction}

\section{Contact Tracing}
Since the beginning of the COVID-19 pandemic, there has been a copious amount
of research in mobile contact tracing solutions, most notably being the joint
effort by Apple and Google \cite{AppleGoogle}. External reviews and surveys
provide extensive comparison of existing solutions through the lenses of
privacy, security, ethics, adversarial models, data management, scalability,
interoperability, and more. References \cite{Ahmed2020} and \cite{Martin2020}
provide thorough reviews of existing mobile contact tracing solutions with
discussion of the techniques, privacy, security, and adversarial models. The
former offers additional detail on the system architecture (i.e., centralized,
decentralized, and hybrid), data management, and user concerns of existing
solutions. Other notable reviews with similar discussion include \cite{Wen2020,
Raskar2020, Cho2020, Dar2020, Lucivero2020}. Reference \cite{Kuhn2021} provides
a formal framework for defining aspects of privacy for proximity-based contact
tracing.

A number of online surveys have been conducted that examine user preferences
of different aspects of contact tracing \cite{Simko2020, Altmann2020, Li2020}.
A common finding across these surveys is that privacy and security continue to
be of top concern for users, but contains some interesting nuance. For example,
\cite{Altmann2020} surveyed over 10,000 individuals and found that there was
over a 60-percent willingness to install a contact tracing mobile application.
In a longitudinal study, \cite{Simko2020} found that user preferences regarding
privacy were stable over time. Moreover, they found that users had fewer
privacy concerns for proximity-based contact tracing, in comparison to
location-based contact tracing, but that there was security concerns for
proximity-based tracking. Contrary to much of the developed techniques that
emphasize a decentralized approach, \cite{Li2020} observed that mobile contact
tracing applications that implement a centralized design are significantly more
likely to be installed at the country level. Additionally, they found that
individuals are generally more comfortable with their location data and
identity information accessible to health-, state-, and federal-level
authorities, compared to application developers, and the general
public.

\section{ShareTrace}
Two prior works on ShareTrace are available \cite{Ayday2020, Ayday2021}. The
former is the original white paper, which focuses on the motivation, design,
and engineering details. Exlusive to \cite{Ayday2020} is a discussion on
privacy, network roaming, protocol interoperability, and the usage of
geolocation data. Furthermore, it includes more detail on the system model and
data flow than \cite{Ayday2021}. The latter work formalizes risk propagation in
a centralized setting and compares its efficacy to the Apple-Google framework
\cite{AppleGoogle}. A concise reiteration of the system model and deployment
model is also provided.

This work is not intended to be a complete reference of ShareTrace. While many
of the concepts will be reiterated in this work, it is assumed that the reader
has read \cite{Ayday2020, Ayday2021}. This work offers the following
contributions to the ShareTrace project and, more generally, to the development
of next-generation applications that utilize self-soverign technologies to
empower individuals take ownership of their personal data:
	\begin{itemize}
		\item a distributed, asynchronous formulation of risk propagation that utilizes the actor model to achieve scalable performance;
		\item a novel form of reachability that accounts for the dynamics of message
passing on a temporal network; and
		\item a serverless architecture for privacy-preserving computing.
	\end{itemize}

ShareTrace is a digital contact-tracing solution that provides individuals
their risk of infection. The two main advantages that ShareTrace offers over
other digital contact-tracing approaches is its use of personal data stores
that provide privacy by architectual design and its use of the contact network
to infer the infection risk from indirect contact. While this thesis covers
many aspects of ShareTrace, it is not comprehensive. The focus of this work
is

Designing, implementing, and analyzing the distributed extension of risk
propagation as first proposed in [CITE]
Presenting a search algorithm for extracting contacts from spatio-temporal
data
Contextualizing risk propagation as a novel usage of a temporal graph

%%%%%%% Old
ShareTrace is a privacy-preserving contact-tracing solution \cite{Ayday2021}.
Unlike other approaches that rely on device proximity to detect human
interaction, ShareTrace executes iterative message passing on a factor graph to
estimate a user's marginal posterior probability of infection (MPPI). To
indicate its similarity to belief propagation, we refer to the ShareTrace
algorithm as \emph{risk propagation}. By considering both direct and indirect
contact, \cite{Ayday2021} demonstrates that risk propagation is more effective
than other proximity-based methods that only consider former.

Building upon the efforts by \cite{Ayday2021}, we provide an efficient and
scalable formulation of risk
propagation\footnote{\url{https://github.com/share-trace}} that utilizes
asynchronous, concurrent message passing on a temporal graph \cite{Holme2012,
Holme2015}. While message passing has been studied under specific
epidemiological models \cite{Karrer2010, Li2021}, our formulation allows us to
contextualize risk propagation as a novel usage of a temporal graph that does
not require such assumptions to infer the transmission of disease. As a result,
we introduce a form of reachability that can uniquely characterize the dynamics
of message passing on a temporal graph. Our formulation of risk propagation
aligns with its distributed extension, as introduced by \cite{Ayday2021}, which
has connections to the actor model of concurrent computing \cite{Baker1977,
Agha1986} and the ``think-like-a-vertex'' model of graph algorithms
\cite{McCune2015}.

Our evaluation aims to (1) describe the efficiency of risk propagation on both
synthetic and real-world temporal graphs; (2) validate the accuracy of our new
form of reachability on both synthetic and real-world graphs; (3) and briefly
quantify the scalability of this implementation of risk propagation on
synthetic graphs. To keep the scope of this work focused, we defer to
\cite{Ayday2021} on the privacy and security aspects of ShareTrace.

\section{Actor Model}
% Informal description of concepts and terminology (no formal semantics)
% Properties and laws
% Actor model vs Akka
% Why Akka?
% Software agents