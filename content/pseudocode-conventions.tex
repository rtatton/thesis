\chapter{Pseudocode Conventions}\label{ch:pseudocode-conventions}

\newcommand{\vSomeObject}{\var{o}}
\newcommand{\vSomeAttribute}{\var{a}}
\newcommand{\aSomeObjectAttribute}{\attr{\vSomeObject}{\vSomeAttribute}}

Pseudocode conventions are mostly consistent with \citet{Cormen2022}.
%
\begin{itemize}
    \item Indentation indicates block structure.
    \item Looping and conditional constructs have similar interpretations to those in standard programming languages.
    \item Composite data types are represented as \define{objects}. Accessing an \define{atttribute} $\vSomeAttribute$ of an object $\vSomeObject$ is denoted
    $\aSomeObjectAttribute$. A variable representing an object is a \define{pointer} or \define{reference} to the data representing the object. The special value
    $\nil$ refers to the absence of an object.
    \item Parameters are passed to a procedure \emph{by value}: the ``procedure receives its own copy of the parameters, and if it assigns a value to a parameters, the change is \emph{not} seen by the calling procedure. When objects are passed, the pointer to the data representing the object is copied, but the attributes of the object are not.'' Thus, attribute assignment ``is visible if the calling procedure has a pointer to the same object.''
    \item A {\Return} statement ``immediately transfers control back to the point of call in the calling procedure.''
    \item Boolean operators {\AND} and {\OR} are \define{short circuiting}.
\end{itemize}
%
The following conventions are specific to this work.
%
\begin{itemize}
    \item Object attributes may be defined \emph{dynamically} in a procedure.
    \item Variables are local to the given procedure, but parameters are global.
    \item The ``$\assign$'' symbol is used to denote assignment, instead of ``$=$''.
    \item The ``$\equals$'' symbol is used to denote equality, instead of ``$==$'', which is consistent with the use of ``$\notEquals$'' to denote inequality.
    \item The ``$\in$'' symbol is used in \textbf{for} loops when iterating over a collection.
    \item Set-builder notation $\setBuilder{\vSomeElement \in \vSomeSet}{\cPredicate[\vSomeElement]}$ is used to create a subset of a collection $\vSomeSet$ in place of constructing an explicit data structure.
\end{itemize}