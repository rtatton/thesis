\section{Asynchronous Risk Propagation}\label{sec:asynchronous}

While straightforward to specify, \cRiskPropagation{}, is not a viable implementation for real-world application, because it is an \define{offline algorithm} that requires all individuals' recent contacts and risk scores to run. As \citet{Ayday2021} note, the centralization of health and contact data is not privacy preserving. An offline design is also computational inefficient and risks human safety. Specifically, most exposure scores may not change across invocations of \cRiskPropagation{}, which implies communication overhead and computational redundancy. As a mitigation, \citet{Ayday2020} suggest running \cRiskPropagation{} only once or twice daily. However, this causes substantial delay in reporting to individuals their exposure scores; and in the face of a pandemic, timely information is essential for individual and collective health.

To address the privacy limitations of \cRiskPropagation{}, \citet{Ayday2021} propose decentralizing the factor graph such that the processing entity (e.g., mobile device or ``personal cloud'') associated with \indexed{i}{individual} maintains the state of \indexed{i}{variable node} and its neighboring factor nodes. But for real-world application, the proposal leaves key questions unanswered:
%
\begin{enumerate}
  \item Is message passing synchronous or asynchronous?
  \item How does message passing terminate?
  \item Are any optimizations used to reduce communication cost?
  \item How do processing entities exchange messages over the network?
\end{enumerate}
%
Key observations:

- Asynchronous message passing amongst stateful entities is the essence of the actor model, so we can use it to describe risk propagation as an online algorithm
- By applying one-mode projection, the factor graph is equivalent to a contact network. Thus, we can extend the concepts of reachability in temporal network to account for message-passing semantics and measure the communication complexity of risk propagation.

The only purpose of a factor node is to compute and relay messages between variable vertices. Thus, one-mode projection onto the variable vertices can be applied such that variable vertices $\vVariable[i],\vVariable[j] \in \vVariables$ are adjacent if the factor node $\vFactor[ij] \in \vFactors$ exists \cite{Zhou2007}. \Cref{fig:projected} shows the modified topology.
%
\begin{figure}[htbp]
\centering
\begin{tikzpicture}[ampersand replacement=\&]
    \matrix[row sep=7em, column sep=2em] {
      \node[latent, minimum size=2em] (v1) {$\vVariable[1]$}; \&\&\&
      \node[latent, minimum size=2em] (v2) {$\vVariable[2]$}; \&\&\&
      \node[latent, minimum size=2em] (v3) {$\vVariable[3]$}; \\
    };
    \factor[minimum size=1em, above= of v1] {f12_1} {above:$\vFactor[12]$} {} {};
    \factor[minimum size=1em, above= of v2, xshift=-3em] {f12_2} {above:$\vFactor[12]$} {} {};
    \factor[minimum size=1em, above= of v2, xshift=3em] {f23_2} {above:$\vFactor[23]$} {} {};
    \factor[minimum size=1em, above= of v3] {f23_3} {above:$\vFactor[23]$} {} {};
    
    \plate{p1} {(v1)(f12_1)(f12_1-caption)} {};
    \plate{p2} {(v2)(f12_2)(f23_2)(f12_2-caption)(f23_2-caption)} {};
    \plate{p3} {(v3)(f23_3)(f23_3-caption)} {};
    
    \edge[-] {v1} {f12_1};
    \edge[-] {v2} {f12_2};
    \edge[-] {v2} {f23_2};
    \edge[-] {v3} {f23_3};
    \edge[-] {p1} {p2};
    \edge[-] {p2} {p3};
  \end{tikzpicture}
\caption[One-mode projection of a factor graph]{One-mode projection onto the variable nodes in \Cref{fig:factor-graph}.}
\label{fig:projected}
\end{figure}

To send a message to variable node $\vVariable[i]$, variable node $\vVariable[j]$ applies the computation associated with the factor node $\vFactor[ij]$. This modification differs from the distributed extension of risk propagation \cite{Ayday2021} in that we do not create duplicate factor vertices and messages in each user's PDS. By storing the contact time between users on the edge incident to their variable vertices, this modified topology is identical to the \emph{contact-sequence} representation of a \emph{contact network}, a kind of \emph{time-varying} or \emph{temporal network} in which a node represents a person and an edge indicates that two persons came in contact:

\begin{equation}
    \setBuilder{(u, v, t)}{u, v \in \vVertices; u \neq v; t \in \naturals}, \label{eq:contact-seq}
\end{equation}
%
where a triple $(u, v, t)$ is called a \emph{contact} \cite{Holme2012}. Specific to risk propagation, $t$ is the time at which users $u$ and $v$ \emph{most recently} came in contact (see \Cref{sec:asynchronous}).

The usage of a temporal network in this work differs from its typical usage in epidemiology which focuses on modeling and analyzing the spreading dynamics of disease \cite{Riolo2001, Danon2011, Lokhov2014, Craft2015, Pastor-Satorras2015, Koher2019, Zino2021}. In contrast, this work uses a temporal network to infer a user's MPPI. As a result, \Cref{sec:reachability} extends temporal reachability to account for both the message-passing semantics and temporal dynamics of the network. As noted by \cite{Holme2012}, the transmission graph provided by \cite{Riolo2001} ``cannot handle edges where one node manages to not catch the disease.'' Notably, the usage of a temporal network in this work allows for such cases by modeling the possibility of infection as a continuous outcome.
% TODO
Factor graphs are useful for decomposing complex probability distributions and allowing for efficient inference algorithms.

However, as with risk propagation, and generally any application of a factor graph in which the variable vertices represent entities of interest (i.e., of which the marginal probability of a variable is desired), applying one-mode projection is a .

\subsection{ShareTrace Actor System}

As a distributed algorithm, risk propagation is specified from the perspective of an \emph{actor}. Some variation exists on exactly how actor behavior is defined \cite{Agha1985, Koster2016}. Perhaps the simplest definition is that the \emph{behavior of an actor} is both its \emph{interface} (i.e., the types of messages it can receive) and \emph{state} (i.e., the internal data it uses to process messages) \cite{Koster2016}. An \emph{actor system}\footnote{This is technically referred to as an \emph{actor system configuration}.} is defined as the set of actors it contains and the set of unprocessed messages\footnote{Formally, a \emph{message} is called a \emph{task} and is defined by a \emph{tag}, a unique identifier; a \emph{target}, the mail address to which the message is delivered; and a \emph{communication}, the message content \cite{Agha1985}.} in the actor mailboxes. An expanded definition of an actor system also includes a \emph{local states function} that maps mail addresses to behaviors, the set of \emph{receptionist actors} that can receive communication that is external to the actor system, and the set of \emph{external actors} that exist outside of the actor system \cite{Agha1985}. Practically, a local states function is unnecessary to specify, so the narrower definition of an actor system is used. The remainder of this section describes the components of the ShareTrace actor system.

\subsection{Actor Behavior}\label{sec:actor-behavior}

Each user corresponds to an actor that participates in the message-passing protocol of risk propagation. Herein, the user of an actor will only be referred to as an \emph{actor}. The following variant of the concurrent, object-oriented actor model is assumed to define actor behavior \cite{Agha1990}.
%
\begin{itemize}
	\item An actor follows the \emph{active object pattern} \cite{Lavender1996, Koster2016} and the \emph{Isolated Turn Principle} \cite{Koster2016}. Specifically, the state change of an actor is carried out by instance- variable assignment, instead of the canonical \cBecome{} primitive that provides a functional construct for pipelining actor behavior replacement \cite{Agha1985, Agha1990}. The interface of a user actor is fixed in risk propagation, so the more general semantics of \cBecome{} is unnecessary.
	\item The term ``name'' \cite{Hewitt1977, Agha1985} is preferred over ``mail address'' \cite{Agha1985, Agha1990} to refer to the sender of a message. Generally, the mail address that is included in a message need not correspond to the actor that sent it. Risk propagation, however, requires this actor is identified in a risk score message. Therefore, to emphasize this requirement, ``name'' is used to refer to both the identity of an actor and its mail address.
	\item An actor is allowed to include a loop with finite iteration in its behavior definition; this is traditionally prohibited in the actor model \cite{Agha1985, Agha1990}.
	\item The behavior definition is implied by all procedures that take as input an actor.
\end{itemize}
%
The \cCreateActor{} operation initializes an actor, which is equivalent to the \emph{new expression} \cite{Agha1985} or \cCreate{} primitive \cite{Agha1990} with the exception that it only specifies the attributes (i.e., state) of an actor. As mentioned earlier, the behavior description of an actor is implied by the procedures that require an actor as input. An actor $\vActor$ has the following attributes.
%
\begin{itemize}
	\item $\aActorName$: the actor's name; used by other actors to communicate with it \citep{Hewitt1977, Agha1985}.
	\item $\aActorContacts$: a dictionary (see \Cref{ch:data-structures}) that maps an actor's name to a time. That is, if \indexed{i}{individual} comes in contact with \indexed{j}{individual}, then $\aContacts{\vActor_i}$ contains $\aName{\vActor_j}$ and the time of contact. This is an extension of the concept of \emph{acquaintances} \citep{Hewitt1977, Agha1985}.
	\item $\aActorScores$: a dictionary that maps a time interval to a risk score; used to tolerate synchronization delays between a user's device and actor (see \Cref{sec:caching}).
	\item $\aActorExposure$: the actor's exposure score. This attribute is either a symptom score, a risk score sent by another actor, or the null risk score (see \cNullRiskScore{}).
\end{itemize}
%
\begin{function}{\nCreateActor}
    \State $\aActorName \assign \cGenerateName$
    \State $\aActorContacts \assign \emptyset$
    \State $\aActorScores \assign \emptyset$
    \State $\aActorExposure \assign \cNullRiskScore[\vActor]$
    \State \Return $\vActor$
\end{function}
%
\begin{function}{\nNullRiskScore}[\vActor]
    \State $\aScoreValue \assign 0$
    \State $\aScoreTime \assign 0$
    \State $\aScoreSender \assign \aActorName$
    \State \Return $\vScore$
\end{function}
%
Risk scores and contacts have finite relevance, which is parametrized by a \emph{liveness} or \emph{relevance duration} $\pScoreExpiry, \pContactExpiry > 0$, respectively. The relevance of risk scores and contacts is important, because it influences how actors pass messages. For example, actors do not send irrelevant risk scores or relevant risk scores to irrelevant contacts. The \emph{time-to-live} (TTL) of a risk score or contact is the remaining duration of its relevance. In the following operations, $\aScoreTime$ denotes the time at which the risk score was originally computed, $\aContactTime$ is the contact time, and $\vRefTime$ is the current time.
%
\begin{function}{\nRiskScoreTtl}[\vScore]
    \State \Return $\pScoreExpiry - (\vRefTime - \aScoreTime)$
\end{function}
\begin{function}{\nContactTtl}[\vContact]
    \State \Return $\pContactExpiry - (\vRefTime - \aContactTime)$
\end{function}
%
The interface of a user actor is defined by two types of messages: contact messages and risk score messages. A \emph{contact message} $\vContact$ contains the name $\aContactName$ of the actor whose user was contacted and the contact time $\aContactTime$. A \emph{risk score message} $\vScore$ is simply a risk score along with the actor's name $\aScoreSender$ that sent it. A risk score previously defined as the ordered pair $(\vScore, t)$ (see \Cref{sec:asynchronous}) is represented as the attributes $\vScore$ and $\aScoreTime$. The following sections discuss how a contact message and risk-message are processed by an actor.

\subsubsection{Contacts}

There are two ways in which a user actor can receive a contact message. The first, technically correct approach is for a receptionist actor to mediate the communication between the user actor and the PDS so that the user actor can retrieve its user's contacts. The second approach is to relax this formality and allow the user actor to communicate with the PDS directly\footnote{If the PDS itself is an actor, then a push-oriented dataflow could be implemented, where the user actor receives contact messages (and symptom-score messages). This would be more efficient and timely than a pull-oriented dataflow in which the PDS is a passive data store that requires the user actor or receptionist to poll it for new data.}.

The \cHandleContact operation defines how a user actor processes a contact message. A contact is assumed to have finite relevance which is parametrized by the \emph{contact time-to-live} $\pContactExpiry > 0$. A contact whose contact time occurred no longer than $\pContactExpiry$ time ago is said to be \emph{alive}. Thus, a user actor only adds a contact if it is alive.
%
\begin{function}{\nHandleContact}[\vActor, \vContact]
  \If{$\cContactTtl[c] > 0$}
    \State $\aContactKey \assign \aContactName$
    \State $\cCacheInsert[\aActorContacts, \vContact]$
    \State $\vScore \assign \cCacheMaxOlderThan[\aActorScores, \aContactTime + \pTimeBuffer]$
    \State $\cApplyRiskScore[\vActor, \vContact, \vScore]$
  \EndIf
\end{function}
%
Regardless of whether the contact is alive, the user actor attempts to send a risk score message that is derived from its current exposure score or a cached risk score (see \Cref{sec:caching}):
%
\begin{function}{\nApplyRiskScore}[\vActor, \vContact, \vScore]
  \State $\cRefreshSendThreshold[\vActor, \vContact]$
  \If{$\cShouldContactReceive[\vContact, \vScore]$}
    \State $\aNewScoreValue \assign \aScoreValue \cdot \pTransmissionRate$
    \State $\aContactBuffered \assign \vNewScore$
    \State $\cSetSendThreshold[\vContact, \vNewScore]$
  \EndIf
\end{function}
%
\begin{function}{\nShouldContactReceive}[\vContact, \vScore]
    \State \Return $\aContactThresholdValue < \aScoreValue$
    \Statex $\AND \aContactTime + \pTimeBuffer > \aScoreTime$
    \Statex $\AND \aContactName \notEquals \aScoreSender$
\end{function}
\begin{function}{\nSetSendThreshold}[\vContact, \vScore]
    \State $\aNewScoreValue \assign \pSendCoefficient \cdot \aScoreValue$
    \State $\aContactThreshold \assign \vNewScore$
\end{function}
%
\begin{function}{\nRefreshSendThreshold}[\vActor, \vContact]
    \If{$\aContactThresholdValue > 0 \AND \cRiskScoreTtl[\aContactThreshold] \leq 0$}
        \State $\vScore \assign \cCacheMaxOlderThan[\aActorScores, \aContactTime + \pTimeBuffer]$
        \State $\aNewScoreValue \assign \aScoreValue \cdot \pTransmissionRate$
        \State $\cSetSendThreshold[\vContact, \vNewScore]$    
    \EndIf
\end{function}
%
\begin{function}{\nHandleFlushTimeout}[\vActor]
  \ForEach{$\vContact \in \aActorContacts$}
    \If{$\aContactBuffered \notEquals \nil$}
      \State $\cSend[\aContactName, \aContactBuffered]$
      \State $\aContactBuffered \assign \nil$
    \EndIf
  \EndFor
  \State $\cCacheRefresh[\aActorContacts]$
\end{function}
%
Like contacts, each risk score is assumed to have finite relevance that is parametrized by the \emph{score time-to-live} $\pScoreExpiry > 0$ and evaluated by the operation \Call{Is-Score-Alive}{}. To send an actor's current exposure score, the contact must be sufficiently recent. It is assumed that risk scores computed after a contact occurred have no effect on the user's exposure score.

To account for the disease incubation period, a delay in reporting symptoms, or a delay in establishing actor communication, a time buffer $\pTimeBuffer \geq 0$ is considered. That is, a risk score is not sent to a contact if \Call{Is-Contact-Recent}{} returns \false.

The \Call{Transmitted}{} operation is used to generate risk scores that are sent to other actors. It is assumed that contact only implies an incomplete transmission of risk between users. Thus, when sending a risk score to another actor, the value of the risk score is scaled by the \emph{transmission rate} $\pTransmissionRate \in (0, 1)$. Notice that the time of the risk score is left unchanged; the act of sending a risk score message is independent of when the risk score was first computed.

%If line \ref{step:send-current} of \cSendCurrentOrCached[] evaluates to \false, then the actor attempts to retrieve the maximum cached risk score message based on the buffered contact time. If such a message exists and is alive, a risk score message is derived and sent to the contact. The \cSend[] operation follows the semantics of the \Call{Send-To}{} primitive \cite{Agha1985, Agha1990}.

\subsubsection{Risk Scores}

Upon receiving a risk score message, an actor executes the following operation.
%
\begin{function}{\nHandleRiskScore}[\vActor, \vScore]
  \If{$\cRiskScoreTtl[\vScore] > 0$}
    \State $\aScoreKey \assign [\aScoreTime, \aScoreTime + \pScoreExpiry)$
    \State $\cCacheInsert[\aActorScores, \vScore]$
    \State $\cUpdateExposureScore[\vActor, \vScore]$
    \ForEach{$\vContact \in \aActorContacts$}
      \State $\cApplyRiskScore[\vActor, \vContact, \vScore]$
    \EndFor
  \EndIf
\end{function}
%
\begin{function}{\nUpdateExposureScore}[\vActor, \vScore]
  \If{$\aActorExposureValue < \aScoreValue$}
    \State $\aActorExposure \assign \vScore$
  \ElsIf{$\cRiskScoreTtl[\aActorExposure] \leq 0$}
    \State $\aActorExposure \assign \cCacheMax[\aActorScores]$
  \EndIf
\end{function}
%
The \Call{Update-Actor}{} operation is responsible for updating an actor's state, based on a received risk score message. Specifically, it stores the message inside the actor's interval cache $\aActorScores$, assigns the actor a new exposure score and send coefficient (discussed below) if the received risk score value exceeds that of the current exposure score, and removes expired contacts.

%In previous work \cite{Ayday2021}, risk propagation assumes synchronous message passing, so the notion of an iteration or inter-iteration difference threshold can be used as stopping conditions. However, as a streaming algorithm that relies on asynchronous message passing, such stopping criteria are unnatural. Instead, the following heuristic is applied and empirically optimized to minimize accuracy loss and maximize efficiency. Let $\pSendCoefficient > 0$ be the \define{send coefficient} such that an actor only sends a risk score message if its value exceeds the actor's \emph{send threshold} $\aContactThreshold$ (line \ref{step:send-condition} in \cPropagate).

Assuming a finite number of actors, any positive send coefficient $\pSendCoefficient$ guarantees that a risk score message will be propagated a finite number of times. Because the value of a risk score that is sent to another actor is scaled by the transmission rate $\pTransmissionRate$, its value exponentially decreases as it propagates away from the source actor with a rate constant $\log \pTransmissionRate$.

%As with the \cSendCurrentOrCached[] operation, a risk score message must be alive and relatively recent for it to be propagated. As in previous work \cite{Ayday2021}, factor marginalization is achieved by not propagating the received message to the actor who sent it. The logic of \cPropagate[] differs, however, in two ways. First, it is possible that no message is not propagated to a contact. The intent of sending a risk score message is to update the exposure score of other actors. However, previous work \cite{Ayday2021} required that a ``null'' message with a risk score value of 0 is sent. Sending such ineffective messages incurs additional communication overhead. The second difference is that only the \emph{most recent} contact time is used to determine if a message should be propagated to a contact. Contact times determine what messages are relevant. Given two contact times $t_1, t_2$ such that $t_1 \leq t_2$, then any risk score with time $t \leq t_1 + \pTimeBuffer$ also satisfies $t \leq t_2 + \pTimeBuffer$. Thus, storing multiple contact times is unnecessary.
%
%\subsubsection{Risk Score Caching}\label{sec:caching}
%
%For two actors to communicate, each must have the other actor in their contacts (see \Cref{sec:actor-behavior}). Recall that an actor must retrieve these contacts from the user's PDS, which subsequently requires synchronization with the user's mobile device (see \Cref{fig:actor-dataflow}). While the user's device can locally store contacts from proximal devices and symptoms of the user, an internet connection is needed to synchronize with the PDS and thus the user's actor. Therefore, it is not only possible but a reality that the user's mobile device and actor will not always be synchronized.
%
%In the best case, this ``lag'' may only be a few seconds; in the worst case, the user's device is offline for several days. If $\delta_i$ ($\delta_j$) is the delay between when the device of user $i$ (ref. $j$) records a contact with user $j$ (ref. $i$) and when its actor receives the corresponding contact message, then the delay between when actors $\vActor_i$ and $\vActor_j$ can communicate bidirectionally and when the contact actually occurred is $\delta_{ij} = \max(\delta_i, \delta_j)$. Such dissonance between the ``true'' state of the world (i.e., when users actually came in contact) and that known to the network of actors could impact the accuracy of risk propagation, which assumes such delays are nonexistent. To address this issue, each actor maintains a cache of received risk score messages such that it can still send a message that reflects its previous state to a contact that was significantly delayed.
%
%An \emph{interval cache} $\vCache$ is a dictionary that maps a finite time interval (key) to a data element (value). In a typical cache, the \emph{time-to-live} (TTL) of an element is a fixed duration after which the element is removed or \emph{evicted}. In an interval cache, however, the TTL of an element is determined by its associated interval and the current time. That is, an interval cache is like a series of sliding windows, where each window corresponds to an interval that can hold a single element. Thus, the TTL of an element is the duration between the start of its interval and the start of the earliest interval in the cache.
%
%An interval cache maintains $\pCacheNumIntervals$ contiguous, half-closed (start-inclusive) intervals, each of duration $\pDuration$. An interval cache contains $\pCacheNumLookAhead$ \emph{look-ahead intervals} and $\pCacheNumLookBack$ \emph{look-back intervals} such that $\pCacheNumIntervals = \pCacheNumLookBack + \pCacheNumLookAhead$. Look- ahead (resp. look-back) intervals allow elements to be associated with intervals whose start times are later (resp. earlier) than the current time $t$. The \emph{look-back duration} $\pCacheLookBack$ and \emph{look-ahead duration} $\pCacheLookAhead$ are defined as
%%
%\begin{align*}
%	\pCacheLookBack &= \pCacheNumLookBack \cdot \pDuration \\
%	\pCacheLookAhead &= \pCacheNumLookAhead \cdot \pDuration .
%\end{align*}
%The \emph{range} of the interval cache is $[\aCacheStart, \aCacheEnd)$, where
%%
%\begin{align*}
%	\aCacheStart &= \aCacheRefresh - \pCacheLookBack \\
%	\aCacheEnd &= \aCacheRefresh + \pCacheLookAhead,
%\end{align*}
%%
%and $\aCacheRefresh$ is the time at which the cache was last refreshed.
%
%An interval cache is a ``live'' data structure, so the range must be updated periodically to reflect the advancement of time. Furthermore, intervals and their associated elements that are no longer contained in the range must be evicted. This process of updating the range and evicting cached elements is called \emph{refreshing the cache}. The \emph{refresh period} $\pCacheRefreshPeriod > 0$ of an interval cache is the duration until the range is updated, based on the current time $t$. Depending on the interval duration $\pDuration$ and the nature of the data that is being cached, the refresh period may be on the order of seconds or days. To recognize when a refresh is necessary, the cache maintains the attribute $\aCacheRefresh$, which is the time of the previous refresh. The operation \cCacheRefresh updates the range if at least $\pCacheRefreshPeriod$ time has elapsed since the previous refresh and then removes all expired elements.
%%
%\begin{function}{\nCacheRefresh}[\vCache]
%    \If{$\vRefTime - \aCacheRefresh > \pCacheRefreshPeriod$}
%    	\State $\aCacheStart \assign \vRefTime - \pCacheLookBack$
%    	\State $\aCacheEnd \assign \vRefTime + \pCacheLookAhead$
%    	\State $\aCacheRefresh \assign \vRefTime$
%    	\ForEach{$\vCacheItem \in \vCache$}
%    		\If{$\aCacheItemKey < \aCacheStart$}
%    			\State $\cDelete[\vCache, \vCacheItem]$
%    		\EndIf
%    	\EndFor
%    \EndIf
%\end{function}
%
%The operation \cCacheInsert refreshes the cache, if necessary, and merges into the cache the element pointed to by $\vCacheItem$ if its timestamp $\aCacheItemTime$ is in the range. Keys in the interval cache are interval start times and are lazily computed (line \ref{step:key}) to avoid storing intervals with no associated element. By not storing all intervals explicitly, the interval cache only achieves $O(\pCacheNumIntervals)$ space complexity when each interval has an associated element. The \cMerge operation (line \ref{step:merge}) can be as trivial as replacing the existing value. For risk propagation, the interval cache associates with each interval the newest risk score of maximum value.
%%
%\begin{function}{\nCacheKey}[\vCache, \vCacheItem]
%    \State \Return $\aCacheStart + \floor{\frac{\aCacheItemTime - \aCacheStart}{\pDuration}} \cdot \pDuration$
%\end{function}
%%
%\begin{function}{Cache-Insert}[\vCache, \vCacheItem]
%    \If{$\aCacheItemTime \in [\aCacheStart, \aCacheEnd)$}
%    	\State $\aCacheItemKey \assign \cCacheKey[\vCache, \vCacheItem]$ \label{step:key}
%    	\State $\vOldCacheItem \assign \cSearch[\vCache, \vCacheItem]$
%    	\If{$\vOldCacheItem \equals \nil$}
%    		\State $\vNewCacheItem \assign \vCacheItem$
%    	\Else
%    		\State $\vNewCacheItem \assign \cMerge[\vOldCacheItem, \vCacheItem]$ \label{step:merge}
%    	\EndIf
%    	\State $\cInsert[\vCache, \vNewCacheItem]$
%    \EndIf
%\end{function}
%
%The intention of sending a cached risk score to a contacted user actor is to account for the delay between when the contact occurred and when the actors establish communication. Therefore, the cached risk score that should be sent is that which would have been the current exposure score at the time the users came into contact. That is, each user actor should send the maximum risk score whose cache interval ends at or before the time of contact, accounting for the time buffer $\pTimeBuffer$ (line \ref{step:cache-max} of \cSendCurrentOrCached in \Cref{sec:actor-behavior}). The operation \cCacheMax is used to carry out this query.
%%
%\begin{function}{\nCacheMax}[\vCache, t]
%    \State \Return $\cMaximum[\{\vCacheItem \in \vCache \mid \aCacheItemKey < \cCacheKey[t]\}]$
%\end{function}
%
%An interval cache is implemented by augmenting a hash table \cite[pp. 253--285]{Cormen2009} with the aforementioned attributes and parameters. By using a hash table, the interval cache offers $\Theta(1)$ average-case insert, search, and delete operations. Reference \cite[pp. 348--354]{Cormen2009} implements an \emph{interval tree} by augmenting a red-black tree. However, insert, delete, and search operations on a red-black tree require $\Theta(\log N)$ in the average case.
%