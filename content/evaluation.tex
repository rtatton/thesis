\chapter{Evaluation}\label{ch:evaluation}

\section{Reference Implementation}

A reference implementation of \cref{sec:asynchronous} is available on GitHub\footnote{\url{https://github.com/cwru-xlab/sharetrace-akka}}. Actors are implemented using the Akka toolkit\footnote{\url{https://doc.akka.io/docs/akka/2.8.5/typed/index.html}}, which offers high performance for large-scale actor systems. Experimental results indicate that the reference implementation can reliably handle contact networks with 1 million individuals and 10 million contacts, which makes it ideal for small-scale experiments. In addition to using the Akka toolkit, several other optimizations are implemented:
\begin{itemize}
  \item To reduce the size of event logs and result files, individual actor identifiers follow zero-based numbering and event records are serialized using the Ion format\footnote{\url{https://amazon-ion.github.io/ion-docs}} with shortened field names.
  \item To reduce memory usage, FastUtil\footnote{\url{https://fastutil.di.unimi.it}} data structures are used, including a specialized integer-based JGraphT\footnote{\url{https://jgrapht.org}} graph implementation \citep{Michail2020}. Also, singletons \citep{Gamma1995}, primitive data types, and reference equality are preferred where feasible and do not impact readability.
  \item To reduce runtime and increase throughput, logging is performed asynchronously with Logback\footnote{\url{https://logback.qos.ch/index.html}} and the LMAX Disruptor\footnote{\url{https://lmax-exchange.github.io/disruptor}}.
\end{itemize}

\Cref{fig:arrow-diagram} shows the dependencies among the application components. Contextualizing this implementation with prior implementations of the driver-monitor-worker (DMW) framework (see \cref{sec:dmw-framework}), \class{RiskPropagation} is the driver, \class{Monitor} is the monitor, and \class{User} is the worker. The key difference between this implementation and previous implementations of the DMW framework is that workers are stateful, which is necessary for decentralization.
\begin{figure}[htbp]
\begin{equation*}
  \class{Main} \rightarrow \class{Runner} \rightarrow \class{RiskPropagation} \rightarrow \class{Monitor} \rightarrow \class{User} \rightarrow \class{Contact}
\end{equation*}
\caption[Arrow diagram of the reference implementation]{Arrow diagram of the reference implementation.}
\label{fig:arrow-diagram}
\end{figure}

\Cref{sec:asynchronous} describes the behavior of \class{User} and \class{Contact}. In order to evaluate \class{RiskPropagation}, each \class{User} also logs the following types of \class{UserEvent}:
\begin{itemize}
  \item \class{ContactEvent}: logged when the \class{User} receives an unexpired \class{ContactMessage}; contains the \class{User} identifier, the \class{Contact} identifier, and the contact time.
  \item \class{ReceiveEvent}: logged when the \class{User} receives an unexpired \class{RiskScoreMessage}; contains the \class{User} identifier, the \class{Contact} identifier, and the \class{RiskScoreMessage}.
  \item \class{UpdateEvent}: logged when the \class{User} updates its exposure score; contains the \class{User} identifier, the previous \class{RiskScoreMessage}, and the current \class{RiskScoreMessage}.
  \item \class{LastEvent}: logged when the \class{User} receives a \class{PostStop} Akka signal\footnote{\url{https://doc.akka.io/docs/akka/current/typed/actor-lifecycle.html\#stopping-actors}} after the \class{Monitor} has stopped; contains the \class{User} identifier and the time of logging the last event, besides \class{LastEvent}; used to detect the end time of message passing.
\end{itemize}
For reachability analysis, \class{RiskScoreMessage} contains the identifier of the \class{User} that propagated the message and the identifier of the \class{User} that first sent the message.

\class{Monitor} is an actor that is responsible for transforming the \class{ContactNetwork} into a collection of \class{User} actors and terminating when no \class{UpdateEvent} has occurred for a period of time. As with \class{User} actors, the \class{Monitor} logs several types of \class{LifecycleEvent}, the meanings of which should be self-explanatory:

\begin{multicols}{2}
\begin{itemize}
  \item \class{CreateUsersStart}
  \item \class{CreateUsersEnd}
  \item \class{SendRiskScoresStart}
  \item \class{SendRiskScoresEnd}
  \item \class{SendContactsStart}
  \item \class{SendContactsEnd}
  \item \class{RiskPropagationStart}
  \item \class{RiskPropagationEnd}
\end{itemize}
\end{multicols}

\class{RiskPropagation} logs execution properties, creates an Akka \class{ActorSystem} that creates a \class{Monitor} actor and sends it a \class{RunMessage}, and then waits until the \class{ActorSystem} terminates. Each execution of \class{RiskPropagation} is associated with a unique key that is included in each event record as mapped diagnostic context (MDC)\footnote{\url{https://logback.qos.ch/manual/mdc.html}}.

The \class{Runner} specifies how \class{RiskPropagation} is created and invoked, usually through some combination of statically defined behavior and runtime configuration.

Finally, \class{Main} is the entry point into the application. It is responsible for parsing \class{Context}, \class{Parameters}, and \class{Runner} from configuration and invoking \class{Runner} with \class{Context} and \class{Parameters} inputs. \class{Context} makes application-wide information accessible, such as the system time and user time\footnote{System time is always the wall-clock time and is included in each logged event record. User time is configurable to either be the wall-clock time or fixed at the reference time. The latter ensures that no \class{RiskScoreMessage} or \class{ContactMessage} expires across executions of \class{RiskPropagation}.}, a pseudorandom number generator, \class{Runner} configuration, and loggers. \class{Parameters}, as the name suggests, is a collection of parameters that modify the behavior of the \class{Monitor}, \class{User}, and \class{Contact}.

In order to analyze the logs that were generated during the execution of \class{RiskPropagation}, they are transformed into a tabular dataset as follows:
\begin{enumerate}
  \item Load the execution properties for all executions of \class{RiskPropagation} that are associated with the same configuration.
  \item Process the stream of event records with one \class{EventHandler} per execution of \class{RiskPropagation}.
  \item Collect the results from each \class{EventHandler} and store them in a file.
  \item To analyze different configurations of \class{RiskPropagation}, load multiple result files and augment the results of each \class{RiskPropagation} execution with its execution properties.
  \item Flatten the resulting data structure and store the tabular dataset.
\end{enumerate}

For evaluation, the following event handlers were implemented:

\begin{itemize}
  \item \class{Reachability}: aggregates \class{ReceiveEvent}s that involve a distinct sender and receiver to determine the influence set cardinality, source set cardinality, and message reachability of each \class{User}.
  \item \class{Runtimes}: aggregates \class{LifecycleEvent}s and \class{LastEvent}s to determine the runtime of creating \class{User}s, sending \class{ContactMessage}s, sending \class{RiskScoreMessage}s, message passing, and the overall execution of \class{RiskPropagation}. Message-passing runtime is the time elapsed from the start of sending \class{RiskScoreMessage}s until the last \class{LastEvent}.
  \item \class{UserEventCounts}: aggregates \class{UserEvent}s to determine the frequency of each subtype for each \class{User}.
  \item \class{UserUpdates}: aggregates \class{UpdateEvent}s to determine the new exposure score of each \class{User} and the change in value.
\end{itemize}

\section{Experimental Design}

The following experiments were used to evaluate \class{RiskPropagation}:
% TODO Check the spacing is appropriate
\begin{enumerate}[itemsep=-4ex, ref={Experiment \arabic*}]
  \item How do the distributions of risk scores and contact times affect runtime? \label{item:distributions}
  \item How do the send coefficient and tolerance affect accuracy and efficiency? \label{item:parameters}
  \item How does the contact network topology affect runtime? \label{item:topology}
\end{enumerate}
\labelcref{item:distributions} and \labelcref{item:topology} focus on benchmarking the reference implementation. More advanced simulation-based analysis of ShareTrace with COVI-AgentSim \citep{Gupta2020} is the subject of future work.

While \labelcref{item:topology} explicitly evaluates the impact of contact network topology on runtime, all research questions were assessed using the same random graphs: Barabasi--Albert graphs \citep{Barabasi1999}, Erd\"{o}s--R\'{e}nyi $G_{n,m}$ graphs \citep{Erdos1959}, Watts--Strogatz graphs \citep{Watts1998}, and random regular graphs \citep{Kim2003}. These graphs were selected because they exhibit, to varying extents, aspects of real-world complex networks \citep{Newman2003}, such as contact networks; they are available in the JGraphT library; and they all are parametric, either directly or indirectly, in the size and order of the network. The latter property allowed the effects of the topology to be isolated.

The following describes the parametrization of each type of contact network. Barabasi--Albert graphs are parametrized by the order $n$, the initial order $n_0$, and the increase in size $m_0$ upon each incremental increase in order. The latter two parameters are determined by solving \cref{eq:Barabasi--Albert-optimization}, where $\fracpart(x)$ is the fractional part of a real number $x$.
\begin{argmini}{n_0, m_0}{\fracpart(m_0)}{\protect\label{eq:Barabasi--Albert-optimization}}{}
  \addConstraint{n_0}{\in \intInterval{1}{n - 1}}
  \addConstraint{m_0}{\in \intInterval{1}{n_0}}
  \addConstraint{m_0}{= \frac{2m - n_0 (n_0 - 1)}{2(n - n_0)}}
\end{argmini}
Erd\"{o}s--R\'{e}nyi $G_{n,m}$ graphs are parametrized by the order $n$ and the size $m$. Random regular graphs are parametrized by the order $n$ and, using the degree sum formula, the degree $d = \lfloor 2m / n \rfloor$. Lastly, Watts--Strogatz graphs \citep{Watts1998} are parametrized by the order $n$, the rewiring probability $p$ and the number of nearest neighbors $k = d + (d \bmod 2)$, which must be even.

\Cref{tab:default-parameters} specifies the default parameter values and seed for pseudorandom number generation. \Cref{tab:experiments} specifies the experiment configurations. All experiments used fixed user time. The following sampling process was used to generate risk scores and contact times.  Given the probability density function $f_X$ and the cumulative distribution function $F_X$ of a random variable $X$, sample a value $x \sim f_X$ and evaluate $c \cdot F_X(x)$ for some scalar $c \in \reals$. Risk scores are composite data types, so risk score values and risk score times were sampled independently. Because risk scores are probabilities, $c = 1$ was used to scale the values. When sampling the times of risk scores and contacts, $c = \pScoreExpiry$ and $c = \pContactExpiry$ were used, respectively.

\begin{table}[htbp]
  \centering
  \begin{tabular}{ll}
    \toprule
    Parameter & Default value \\
    \midrule
    Transmission rate, $\pTransmissionRate$ & \num{0.8} \\
    Send coefficient, $\pSendCoefficient$ & \num{1} \\
    Tolerance, $\pTolerance$ & \num{0} \\
    Time buffer, $\pTimeBuffer$ & \qty{2}{days} \\
    Risk score expiry, $\pScoreExpiry$ & \qty{14}{days} \\
    Contact expiry, $\pContactExpiry$ & \qty{14}{days} \\
    Flush timeout & \qty{3}{seconds} \\
    Idle timeout & \qty{1}{minute} \\
    Seed & \num{12345} \\
    \bottomrule
  \end{tabular}
  \caption[Default parameter values for experiments]{Default parameter values for experiments.}
  \label{tab:default-parameters}
\end{table}

\begin{sidewaystable}[htbp]
  \centering
  \renewcommand{\arraystretch}{2}
  \begin{tabular}{lccc}
    \toprule
    Aspect & \labelcref{item:parameters} & \labelcref{item:distributions} & \labelcref{item:topology} \\
    \midrule
    Order $n$ and size $m$ & $\begin{aligned} n &= 10^4 \\ m &= 5 \cdot 10^4 \end{aligned}$ & $\begin{aligned} n &= 10^4 \\ m &= 5 \cdot 10^4 \end{aligned}$ & $\begin{matrix} n \in \setBuilder{10^5x}{x \in \intInterval{1}{10}} \\ \times \\ m \in \setBuilder{10^6x}{x \in \intInterval{1}{10}} \end{matrix}$ \\
    \hline
    Parameters & $\begin{matrix} \pSendCoefficient \in \setBuilder{10^{-1}x}{x \in \intInterval{8}{20}} \\ \pTolerance \in \setBuilder{10^{-3}x}{x \in \intInterval{1}{10}} \end{matrix}$ & Defaults & Defaults \\
    \hline
    Distributions & $\{\text{Uniform}, \text{Standard normal}\}^3$ & $\{\text{Uniform}, \text{Standard normal}\}^3$ & Uniform \\
    \hline
    Repetitions & 5 & 1 burn-in + 5 & 1 burn-in + 5 \\
    \hline
    Networks evaluated & 160 (40 per type) per parameter & 160 (40 per type) & 400 (100 per type) \\
    \bottomrule
  \end{tabular}
  \caption[Experiment configurations]{Experiment configurations. See \cref{tab:default-parameters} for default parameter values. In \labelcref{item:parameters}, the send coefficient and tolerance were evaluated independently. The notation $X^k$ is used to denote the $k$-ary Cartesian power of the set $X$. A ``burn-in'' repetition was used for \labelcref{item:distributions} and \labelcref{item:topology} to avoid measuring the impact of Java class loading.}
  \label{tab:experiments}
\end{sidewaystable}

\section{Results}

\subsection{\labelcref{item:distributions}}

\labelcref{item:distributions} was completed first to determine if it was necessary to evaluate multiple data distributions for \labelcref{item:parameters} and \labelcref{item:topology}. A one-way analysis of variance (ANOVA) can determine if the mean message-passing runtimes associated with different data distributions are statistically different. However, ANOVA assumes the observations are independently sampled, normally distributed, and homoscedastic within groups.

The message-passing runtimes were indeed independently sampled. The Shapiro--Wilk test for normality \cite{Shapiro1965} produces a test statistic of \num{0.672} ($p=\num{2e-17}$). Because $p < \num{0.05}$, there is evidence to reject the null hypothesis that the message-passing runtimes are normally distributed.
%The $p$ values specified in \cref{tab:runtime-normality} for network-specific runtimes also indicate that the runtimes are not normally distributed.
The Fligner--Killeen test \cite{Fligner1976} for homoscedasticity produces a test statistic of \num{2.256} ($p = \num{0.944}$). Given $p > \num{0.05}$, the test does not provide evidence to reject the null hypothesis that the groups have equal variance. Because the independence and homoscedasticity assumptions hold, the nonparametric Kruskal--Wallis test \cite{Kruskal1952} can be used instead of a one-way ANOVA.

The Kruskal--Wallis test results in a test statistic of \num{7.030} ($p=\num{0.426}$). With $p > \num{0.05}$, there is not evidence to reject the null hypothesis that the median message-passing runtimes are statistically different across data distributions. Note, this experiment assumes this statistical significance holds across contact networks of various orders and sizes.

%\begin{table}[htbp]
%\centering
%\begin{tabular}{lSS}
%  \toprule
%  Network type & {Test statistic} & {$p$ value} \\
%  \midrule
%  Barabasi--Albert & 0.646 & 1e-8 \\
%  Erd\"{o}s--R\'{e}nyi & 0.472 & 8e-11 \\
%  Random regular & 0.583 & 2e-9 \\
%  Watts--Strogatz & 0.437 & 3e-11 \\
%  \bottomrule
%\end{tabular}
%\caption[Runtime normality]{Runtime normality.}
%\label{tab:runtime-normality}
%\end{table}