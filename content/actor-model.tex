\chapter{Actor Model}\label{ch:actor-model}

\newcommand{\vEvents}{\vSet{E}}
\newcommand{\vEvent}{e}

The \define{actor model} is a local model of concurrent computing that defines computation as a strict partial ordering of events \citep{Hewitt1977a, Hewitt1977b}. An \define{event} is defined as ``a transition from one local state to another'' \citep{Hewitt1977b}; or, more concretely, ``a \textit{message} arriving at a computational agent called an \textit{actor}.'' The actor that receives the message of an event is called the \define{target} of the event.

Upon receipt, an actor may send messages to itself or other actors, update its local state, or create other actors.

May send messages to other actors (Clinger)
Update its local state (Clinger)
Create other actors

Events caused 

An event may \define{activate} subsequent events

% activate, activator, external event
% precedes ordering, activation ordering, arrival ordering, combined ordering

Ordering laws:

% If an actor sends a message and the target never receives it, does it 

\clearpage

Passing messages between actors is the only means of communication. Thus, control and data flow are inseparable in the actor model \citep{Hewitt1973}.

% TODO - pull from Hewitt1977b regarding name
A minimal specification of an actor includes its \define{name} and \define{behavior}, or how it acts upon receiving a message. \citet{Clinger1981} explains,
%
\begin{displayquote}
  An actor's name is a necessary part of its description because two different actors may have the same behavior. An actor's behavior is a necessary part of its description because the same actor may have different behaviors at different times.
\end{displayquote}
%
\define{acquaintances}, or a finite collection of names of the actors that it knows about \citep{Hewitt1973}.

% De_Koster also includes interface, state, and mailbox as part of an actor definition

% TODO - move this into chapter 2

Thus, the specification of an actor system consists of the following \citep{Hewitt1977a}.

\begin{quote}
\begin{itemize}
  \item Deciding on the natural kinds of actors (objects) to have in the system to be constructed.
  \item Deciding for each kind of actor what kind of messages it should receive.
  \item Deciding for each kind of actor what it should do when it receives each kind of message.
\end{itemize}
\end{quote}

%
% Some variation exists on exactly how actor behavior is defined \citep{Agha1985, Koster2016}. Perhaps the simplest definition is that the \emph{behavior of an actor} is both its \emph{interface} (i.e., the types of messages it can receive) and \emph{state} (i.e., the internal data it uses to process messages) \citep{Koster2016}. An \emph{actor system}\footnote{This is technically referred to as an \emph{actor system configuration}.} is defined as the set of actors it contains and the set of unprocessed messages\footnote{Formally, a \emph{message} is called a \emph{task} and is defined by a \emph{tag}, a unique identifier; a \emph{target}, the mail address to which the message is delivered; and a \emph{communication}, the message content \citep{Agha1985}.} in the actor mailboxes. An expanded definition of an actor system also includes a \emph{local states function} that maps mail addresses to behaviors, the set of \emph{receptionist actors} that can receive communication that is external to the actor system, and the set of \emph{external actors} that exist outside of the actor system \citep{Agha1985}. Practically, a local states function is unnecessary to specify, so the narrower definition of an actor system is used.

%\begin{itemize}
%  \item An actor follows the \define{active object pattern} \citep{Lavender1996, Koster2016} and the \define{Isolated Turn Principle} \citep{Koster2016}. Specifically, the state change of an actor is carried out by instance- variable assignment, instead of the canonical \cBecome{} primitive that provides a functional construct for pipelining actor behavior replacement \citep{Agha1985}. The interface of an actor is fixed in risk propagation, so the more general semantics of \cBecome{} is unnecessary.
%  \item The term ``name'' \citep{Hewitt1977, Agha1985} is preferred over ``mail address'' \citep{Agha1985} to refer to the sender of a message. Generally, the mail address that is included in a message need not correspond to the actor that sent it. Risk propagation, however, requires this actor is identified in a risk score message. Therefore, to emphasize this requirement, ``name'' is used to refer to both the identity of an actor and its mail address.
%  \item An actor is allowed to include a loop with finite iteration in its behavior definition; this is traditionally prohibited in the actor model \citep{Agha1985}.
%  \item The behavior definition is implied by all procedures that take as input an actor.
%\end{itemize}