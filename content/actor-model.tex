\chapter{Actor Model}\label{ch:actor-model}

\newcommand{\vEvents}{\vSet{E}}
\newcommand{\vEvent}{e}

The \define{actor model} is a local model of concurrent computing that defines computation as a strict partial ordering of events \citep{Hewitt1977a, Hewitt1977b}. An \define{event} is defined as ``a transition from one local state to another'' \citep{Hewitt1977b}; or, more concretely, ``a \textit{message} arriving at a computational agent called an \textit{actor}.'' The actor that receives the message of an event is called the \define{target} of the event.

Upon receipt, an actor may send messages to itself or other actors, update its local state, or create other actors.

May send messages to other actors (Clinger)
Update its local state (Clinger)
Create other actors

Events caused 

An event may \define{activate} subsequent events

% activate, activator, external event
% precedes ordering, activation ordering, arrival ordering, combined ordering

Ordering laws:

% If an actor sends a message and the target never receives it, does it 

\clearpage

Passing messages between actors is the only means of communication. Thus, control and data flow are inseparable in the actor model \citep{Hewitt1973}.

% TODO - pull from Hewitt1977b regarding name
A minimal specification of an actor includes its \define{name} and \define{behavior}, or how it acts upon receiving a message. \citet{Clinger1981} explains,
%
\begin{displayquote}
  An actor's name is a necessary part of its description because two different actors may have the same behavior. An actor's behavior is a necessary part of its description because the same actor may have different behaviors at different times.
\end{displayquote}
%
\define{acquaintances}, or a finite collection of names of the actors that it knows about \citep{Hewitt1973}.

% De_Koster also includes interface, state, and mailbox as part of an actor definition

% TODO - move this into chapter 2

Thus, the specification of an actor system consists of the following \citep{Hewitt1977a}.

\begin{quote}
\begin{itemize}
  \item Deciding on the natural kinds of actors (objects) to have in the system to be constructed.
  \item Deciding for each kind of actor what kind of messages it should receive.
  \item Deciding for each kind of actor what it should do when it receives each kind of message.
\end{itemize}
\end{quote}