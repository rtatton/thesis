\chapter{Typographical Conventions}\label{sec:conventions}

% TODO Update mathematical conventions
\section{Mathematics}

Mathematical typesetting follows the guidance of \cite{Dyer}.

\section{Pseudocode}

The pseudocode conventions used in this work mostly follow \cite[pp.
21--24]{Cormen2022}.
\begin{itemize}
    \item Indentation indicates block structure.
    \item Looping and conditional constructs have similar interpretations to
    those in standard programming languages.
    \item Composite data types are represented as \define{objects}. Accessing
    an \define{atttribute} $\var{a}$ of an object $\var{o}$ is denoted
    $\attr{o}{a}$. A variable representing an object is a \emph{pointer} or
    \emph{reference} to the data representing the object. The special value
    $\nil$ refers to the absence of an object.
    \item Parameters are passed to a procedure \emph{by value}. That is, the
    ``procedure receives its own copy of the parameters, and if it assigns a
    value to a parameters, the change is \emph{not} seen by the
    calling procedure. When objects are passed, the pointer to the data
    representing the object is copied, but the object's attributes are not''
    \cite[p. 23]{Cormen2022}. Thus, object attribute assignment ``is visible if
    the calling procedure has a pointer to the same object'' 
    \cite[p. 24]{Cormen2022}.
    \item A \Return statement ``immediately transfers control back to the point
    of call in the calling procedure'' \cite[p. 24]{Cormen2022}.
    \item Boolean operators {\AND} and {\OR} are \emph{short circuiting}.
\end{itemize}
The following conventions are specific to this work.
\begin{itemize}
    \item Object attributes may be defined \emph{dynamically} in a procedure.
    \item Variables are local to the given procedure, but parameters are
    global.
    \item The ``$\assign$'' symbol is used to denote assignment, instead of
    ``$=$''.
    \item The ``$\equals$'' symbol is used to denote equality, instead of
    ``$==$'', which is consistent with the use of ``$\notEquals$'' to denote
    inequality.
    \item The ``$\in$'' symbol is used in \textbf{for} loops when iterating
    over a collection.
    \item Set-builder notation $\{x \in X \mid \cPredicate[x]\}$ is used to
    create a subset of a collection $X$ in place of constructing an explicit
    data structure.
\end{itemize}