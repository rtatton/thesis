\chapter{Conclusions}

%\par Applications like ShareTrace are fundamentally collaborative in that users exchange data amongst each other to achieve an objective or gain personal utility. Maintaining personal data ownership and privacy in this collaborative setting while ensuring architectural scalability and security is an ongoing challenge in the fields of machine learning and cloud computing \cite{Cano2015, Hsieh2017, Jonas2017, Singhvi2017}. Our formulation of risk propagation offers scalability and efficiency and is thus a viable candidate for real-world usage to estimate the spread of infectious diseases. Moreover, message reachability provides researchers and system designers the ability to quantify both the risk of an individual and the effects parameter values have on the efficiency and accuracy of risk propagation.
%
%In future work, we intend to consider mechanisms of establishing decentralized, verifiable communication channels \cite{Abramson2020} as a means to satisfy the collaborative requirements of user-centric applications, such as ShareTrace. Moreover, we shall consider how privacy-preserving mechanisms, such as differential privacy \cite{Dwork2014}, may be utilized in such a setting to minimize the personal risks of widespread data sharing.

%\section{Future Work}
%
%\subsection{Add PDA functionality}
%\par While our current implementation only accounts for user-reported symptoms, but allows for rich integration of other user data streams, such as wearable and mobile health tracking applications, machine-generated biomarkers (e.g., temperature, coughing, heart rate, oxygen saturation level), and electronic health records. This additional information would allow us to provide advanced and personalized recommendations to the user.
%\par It is important to note that because ShareTrace uses location-based contact tracing, it is not currently interoperable with proximity-based approaches. However, users of ShareTrace gain additional personalized risk assessments based on their symptoms and existing conditions. As part of future work, ShareTrace will offer in-app opt-in options for users to share their risks with government agencies, healthcare providers, their employer, and research organizations. Using the HAT Microserver, users can legally and functionally control how their data is shared with these organizations.
%
%\subsection{Extend to distributed architecture}
%\par PDAs currently function as passive data stores. That is, they are not able to communicate with other PDAs. However, given that PDAs can communicate actively, we can formulate risk propagation as a distributed algorithm in which we partition the factor graph amongst all the PDAs. Each PDA would contain a variable node and neighboring factor nodes. To minimize the amount of user information that is transferred between PDAs, we can utilize differential privacy \cite{Cynthia2008}. In a distributed setting, the message-passing aspect of risk propagation would follow the tenets of reactive streams \cite{manifesto214, streams2021}. With such an architecture, we allow for true scalability and further privacy preservation.
%
%\subsection{Validate on diverse cohorts}
%\par Given that a vaccine is now available for the COVID-19 virus, it is unlikely that ShareTrace, at least as a contact tracing application, will be of immediate relevance. However, to fully understand the effectiveness of our approach, we would need to conduct validation studies on diverse cohorts, such as university students, health care workers, and essential workers. This is particularly important because of the intricacies of human behavioral dynamics, which can be difficult to simulate.