\chapter{Conclusion}

% https://writingcenter.fas.harvard.edu/conclusions

% Summary

% Our formulation of risk propagation offers scalability and efficiency and is thus a viable candidate for real-world usage to estimate the spread of infectious diseases. Moreover, message reachability provides researchers and system designers the ability to quantify both the risk of an individual and the effects parameter values have on the efficiency and accuracy of risk propagation.

% Why future work is still needed

In May 2023, the World Health Organization (WHO) declared that COVID-19 is no longer a ``global health emergency'' \citep{Wise2023}. However, as is evident throughout history, the risk posed by emerging pathogens persists \citep{Piret2021, Tabish2022}. Thus, research on effective approaches, such as contact tracing, to preventing and mitigating future outbreaks remains critically important.

% Limitations

% Approach: risk score calculation is not dynamic, personalized
% Approach: risk transmission does not account for modeling techniques
% Approach: susceptible to security and privacy risks
% Evaluation: limited to synthetic networks

% Future work

% Integrate with differential privacy
% Integrate with modeling techniques of risk score calculation and risk transmission \citep{Ferretti2020, Ferretti2024}
% Integrate the strengths of other message-based approaches, such as Ovid
% Integrate data verifiability and integrity with self-sovereign identity technologies (DIDComm Messaging, Verifiable Credentials)

% Evaluate the feasibility of using decentralized technologies for implementation (IPFS, DIDs, Verifiable Credentials)

%Applications like ShareTrace are fundamentally collaborative in that users exchange data amongst each other to achieve an objective or gain personal utility. Maintaining personal data ownership and privacy in this collaborative setting while ensuring architectural scalability and security is an ongoing challenge in the fields of machine learning and cloud computing.

%- Push-based messaging (this work):
%	- Description: individuals maintain exposure score history, send thresholds of contacts, and mailbox identifiers of their contacts. Contacts are sent risk scores that have the possibility of updating their exposure score. Messages are encrypted by the public key of the receiver.
%	- Advantage: only the receiver can decrypt the messages sent to them
%	- Advantage: exposure score history is stored locally
%	- Advantage: exposure score is only sent if its possible to update the exposure score of the receiving contact
%	- Weakness: messages may expire while waiting in the mailbox, which is wasted communication overhead
%	- Weakness: mailboxes are susceptible to spam/DDoS
%	- Weakness: sending messages is correlated with a change in an individual's exposure score or a change in the contact's send threshold, which leaks information about exposure risk of the sender and receiver.
%- Pull-based Pub/Sub (future work):
%	- Description: individuals maintain a list of append-only Pub/Sub topics, topic identifiers of their contacts, public keys associated with their contacts' topics. Individuals maintain the last read message from each contact's topic. If new messages have been published, the individual reads them, updates their state, publishes new messages.
%	- Advantage: write-access to topics are limited to the publisher. This protects individuals from malicious spammers.
%	- Advantage: only non-expired messages are acted upon
%	- Advantage: reading and writing of messages need not correlate to a change in exposure risk
%	- Weakness: exposure score history is stored publicly; anyone with the individual's public key can learn the exposure scores of the individual