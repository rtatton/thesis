\chapter{Data Structures}

\newcommand{\dset}{S}

Let a \define{dynamic set} $\dset$ be a mutable collection of distinct elements, and a \define{dictionary} be a dynamic set that supports insertion, deletion, and membership querying. Each element of $\dset$ is represented as an object $\vCacheItem$ with attributes such that $\aCacheItemKey$ is a unique identifier for the object $\vCacheItem$ \cite[p. 249]{Cormen2022}. The operations \cSearch, \cInsert, \cDelete, \cMinimum, and \cMaximum{} are mostly consistent with \cite[p. 250]{Cormen2022}.
%
\begin{itemize}
    \item $\cSearch[S, k]$ returns a pointer $\vCacheItem$ to an element in the set $\dset$ such that $\aCacheItemKey \equals k$, or $\nil$ if no such element belongs to $\dset$.
    \item $\cInsert[\dset, \vCacheItem]$ adds the element pointed to by $\vCacheItem$ to the set $\dset$.
    \item $\cDelete[\dset, \vCacheItem]$ removes the element pointed to by $\vCacheItem$ from the set $\dset$.
    \item $\cMinimum[\dset]$ and $\cMaximum[\dset]$ return a pointer $\vCacheItem$ to the minimum and maximum element, respectively, of the totally ordered set $\dset$, or $\nil$ if $\dset$ is empty. Reference \cite{Cormen2022} only considers the \emph{key} attribute of the pointers when finding the minimum or maximum element of $\dset$. This work, however, will allow other attributes to be used.
\end{itemize}