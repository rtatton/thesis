\chapter{Data Structures}

\newcommand{\set}{S}

Let a \define{dynamic set} $\set$ be a mutable collection of distinct elements,
and a \define{dictionary} be a dynamic set that supports insertion, deletion,
and membership querying. Each element of $\set$ is represented as an object
$\vCacheItem$ with attributes such that $\aCacheItemKey$ is a unique identifier
for the object $\vCacheItem$ \cite[p. 249]{Cormen2022}. The operations
\cSearch, \cInsert, \cDelete, \cMinimum, and \cMaximum{} are mostly consistent
with \cite[p. 250]{Cormen2022} and are provided below for completeness.
\begin{itemize}
    \item $\cSearch[S, k]$ returns a pointer $\vCacheItem$ to an element in the
    set $\set$ such that $\aCacheItemKey \equals k$, or $\nil$ if no such
    element belongs to $\set$.
    \item $\cInsert[\set, \vCacheItem]$ adds the element pointed to by
    $\vCacheItem$ to the set $\set$.
    \item $\cDelete[\set, \vCacheItem]$ removes the element pointed to by
    $\vCacheItem$ from the set $\set$.
    \item $\cMinimum[\set]$ and $\cMaximum[\set]$ return a pointer
    $\vCacheItem$ to the minimum and maximum element, respectively, of the
    totally ordered set $\set$, or $\nil$ if $\set$ is empty. Reference
    \cite{Cormen2022} only considers the \emph{key} attribute of the pointers
    when finding the minimum or maximum element of $\set$. This work, however,
    will allow other attributes to be used.
\end{itemize}