% -*-latex-*-

\title{SHARETRACE: CONTACT TRACING WITH THE ACTOR MODEL}

\author{RYAN TATTON}
\department{Department of Computer and Data Sciences}

\degree{Master of Science in Computer Science}

\degreemonth{August}
\degreeyear{2022}
\thesisdate{July 31, 2022}

%% By default, the thesis will be copyrighted to MIT.  If you need to copyright
%% the thesis to yourself, just specify the `vi' documentclass option.  If for
%% some reason you want to exactly specify the copyright notice text, you can
%% use the \copyrightnoticetext command.  
% \copyrightnoticetext{\copyright IBM, 1990.  Do not open till Xmas.}

\copyrightnoticetext{}

\supervisor{Erman Ayday}{Assistant Professor}

% This is the department committee chairman, not the thesis committee
% chairman.  You should replace this with your Department's Committee
% Chairman.
\chairman{Vipin Chaudhary}{Chairman, Department Committee on Graduate Theses}

% Make the titlepage based on the above information.  If you need
% something special and can't use the standard form, you can specify
% the exact text of the titlepage yourself.  Put it in a titlepage
% environment and leave blank lines where you want vertical space.
% The spaces will be adjusted to fill the entire page.  The dotted
% lines for the signatures are made with the \signature command.
\maketitle

% The abstractpage environment sets up everything on the page except
% the text itself.  The title and other header material are put at the
% top of the page, and the supervisors are listed at the bottom.  A
% new page is begun both before and after.  Of course, an abstract may
% be more than one page itself.  If you need more control over the
% format of the page, you can use the abstract environment, which puts
% the word "Abstract" at the beginning and single spaces its text.

%% You can either \input (*not* \include) your abstract file, or you can put
%% the text of the abstract directly between the \begin{abstractpage} and
%% \end{abstractpage} commands.

% -*- Mode:TeX -*-
%
% Some departments (e.g. Chemistry) require an additional cover page
% with signatures of the thesis committee.  Please check with your
% thesis advisor or other appropriate person to determine if such a 
% page is required for your thesis.  
%
% If you choose not to use the "titlepage" environment, a \newpage
% commands, and several \vspace{\fill} commands may be necessary to
% achieve the required spacing.  The \signature command is defined in
% the "mitthesis" class
%
% The following sample appears courtesy of Ben Kaduk <kaduk@mit.edu> and
% was used in his June 2012 doctoral thesis in Chemistry. 

\begin{titlepage}
\begin{large}
\textsc{\bfseries Case Western Reserve University}\\
\textsc{\bfseries School of Graduate Studies}

We hereby approve the thesis/dissertation of

\textbf{Ryan Tatton}

candidate for the degree of Masters in Computer Science.

\textbf{Comittee Chair}

Erman Ayday

\textbf{Committee Member}

Youngjin Yoo

\textbf{Committee Member}

Harold Connamacher

\textbf{Committee Member}

Michael Lewicki

\textbf{Date of Defense}

31 July 2022

*We also certify that written approval has been obtained for any proprietary material contained therein.

%\signature{Professor Jianshu Cao}{Chairman, Thesis Committee \\
%   Professor of Chemistry}
%
%\signature{Professor Troy Van Voorhis}{Thesis Supervisor \\
%   Associate Professor of Chemistry}
%
%\signature{Professor Robert W. Field}{Member, Thesis Committee \\
%   Haslam and Dewey Professor of Chemistry}
\end{large}
\end{titlepage}



% First copy: start a new page, and save the page number.
\cleardoublepage
% Uncomment the next line if you do NOT want a page number on your
% abstract and acknowledgments pages.
% \pagestyle{empty}
\setcounter{savepage}{\thepage}
\begin{abstractpage}
\begin{center}
\large

\MakeUppercase{\thesisTitle}

\vspace{0.1in}

\vspace{0.1in}

\normalsize

by

\vspace{0.1in}

\large

\MakeUppercase{Ryan Tatton}

\vspace{0.1in}

\normalsize

\end{center}

% TODO 150 words max
\section*{Abstract}
Proximity-based contact tracing relies on mobile-device interaction to estimate the spread of disease. ShareTrace is one such approach that improves the efficacy of tracking disease spread by considering direct and indirect forms of contact. In this work, we utilize the actor model to provide an efficient and scalable formulation of ShareTrace with asynchronous, concurrent message passing on a temporal contact network. We also introduce message reachability, an extension of temporal reachability that accounts for network topology and message-passing semantics. Our evaluation on both synthetic and real-world contact networks indicates that correct parameter values optimize for algorithmic accuracy and efficiency. In addition, we demonstrate that message reachability can accurately estimate the risk a user poses to their contacts.
\clearpage
\end{abstractpage}

% Additional copy: start a new page, and reset the page number.  This way,
% the second copy of the abstract is not counted as separate pages.
% Uncomment the next 6 lines if you need two copies of the abstract
% page.
% \setcounter{page}{\thesavepage}
% \begin{abstractpage}
% \begin{center}
\large

\MakeUppercase{\thesisTitle}

\vspace{0.1in}

\vspace{0.1in}

\normalsize

by

\vspace{0.1in}

\large

\MakeUppercase{Ryan Tatton}

\vspace{0.1in}

\normalsize

\end{center}

% TODO 150 words max
\section*{Abstract}
Proximity-based contact tracing relies on mobile-device interaction to estimate the spread of disease. ShareTrace is one such approach that improves the efficacy of tracking disease spread by considering direct and indirect forms of contact. In this work, we utilize the actor model to provide an efficient and scalable formulation of ShareTrace with asynchronous, concurrent message passing on a temporal contact network. We also introduce message reachability, an extension of temporal reachability that accounts for network topology and message-passing semantics. Our evaluation on both synthetic and real-world contact networks indicates that correct parameter values optimize for algorithmic accuracy and efficiency. In addition, we demonstrate that message reachability can accurately estimate the risk a user poses to their contacts.
\clearpage
% \end{abstractpage}

\cleardoublepage

\section*{Acknowledgments}

\par This work made use of the High Performance Computing Resource in the Core Facility for Advanced Research Computing at Case Western Reserve University.

%%%%%%%%%%%%%%%%%%%%%%%%%%%%%%%%%%%%%%%%%%%%%%%%%%%%%%%%%%%%%%%%%%%%%%
% -*-latex-*-
